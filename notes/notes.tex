\documentclass{article}
\usepackage{charter}
\usepackage[utf8]{inputenc}
\usepackage{amsmath}
\usepackage{amsfonts}
\usepackage{amssymb}
\usepackage{mathrsfs}
\usepackage{graphicx}
\usepackage[shortlabels]{enumitem}
\usepackage[a4paper, total={7in, 9in}]{geometry}
\usepackage{titlesec}
\usepackage{esint}
\usepackage{amsthm}
\usepackage{tikz}
\usepackage{tocloft}
\usepackage{algorithm}
\usepackage{algorithmic}
\usepackage{cancel}
\usepackage{hyperref}
\hypersetup{
    colorlinks=true,
    linktoc=all,
    linkcolor=blue,
    urlcolor=blue,
}


\newtheorem{theorem}{Theorem}[subsection] % Base counter shared with section
\newtheorem{corollary}[theorem]{Corollary}
\newtheorem{lemma}[theorem]{Lemma}
\newtheorem{definition}[theorem]{Definition}
\newtheorem{proposition}[theorem]{Proposition}
\theoremstyle{definition}
\newtheorem{example}[theorem]{Example}

\theoremstyle{definition}
\newtheorem{remark}[theorem]{Remark}

\renewcommand{\P}{\mathbb{P}}
\newcommand{\Z}{\mathbb{Z}}
\newcommand{\R}{\mathbb{R}}
\newcommand{\Q}{\mathbb{Q}}
\newcommand{\Rn}{\mathbb{R}^{n\times n}}
\newcommand{\Rm}{\mathbb{R}^{m\times n}}
\newcommand{\N}{\mathbb{N}}
\newcommand{\s}{\sigma}
\newcommand{\xs}{x^{\star}}
\newcommand{\tb}{\bar{\theta}}
\renewcommand{\th}{\theta}

\newcommand{\ui}{\overline{\int_{a}^{b}}}
\newcommand{\li}{\underline{\int_{a}^{b}}}

\title{Theory of Finite Elements}
\author{Ansh Desai \\ \href{mailto:adesai@udel.edu}{adesai@udel.edu}}
\date{\today}

\setcounter{tocdepth}{3}
\begin{document}

\maketitle

\tableofcontents

\section{Day 1: Ordinary Differential Equations}
\subsection{Lecture 1: Introduction and Theory}
\subsubsection{Projectile Motion}
Consider the trajectory of a projectile with launch angle $\alpha$ and launch speed $S_0$. Then, the projectile at time $t$ has horizontal and vertical position $x(t),y(t)$ and velocity $v_x(t),v_y(t)$ such that
$$\frac{d}{dt}x(t)=v_x(t),\quad \frac{d}{dt}v_x(t)=0$$
$$\frac{d}{dt}y(t)=v_y(t),\quad \frac{d}{dt}v_y(t)=-g$$
where $g$ is the graviational constant. We precribe the initial conditions $x(0)=x_0$, $y(0)=y_0$, $v_x(0)=v_{x0}$, $v_y(0)=y_{y0}(0)$. This can easily be solved by hand
$$x(t)=x_0+v_{x,0}t,\quad y(t)=y_0+v_{y,0}t-\frac{1}{2}gt^2.$$
The typical associated optimization problem is to find $\alpha$ such that the horizontal distance travelled is maximized, where $v_{x,0}=S_0\cos(\alpha)$, $v_{y,0}=S_0\sin(\alpha)$. Observe that we first want to find $t^*>0$ such that $y(t^*)=0$. Indeed, we have that
$$0=y_0+S_0\sin(\alpha)t-\frac{1}{2}gt^2$$
$$\implies t^*=\frac{S_0\sin\alpha+\sqrt{S_0^2\sin^2\alpha+2gy_0}}{g}.$$
Thus,
$$D(\alpha)=x(t^*)=x_0+S_0\cos\alpha\frac{S_0\sin\alpha+\sqrt{S_0^2\sin^2\alpha+2gy_0}}{g}.$$
A necessary condition for optima is $\frac{d}{d\alpha}D(\alpha)=0$. For $y_0=0$,
$$0=\frac{2S_0^2}{g}(-\sin^2\alpha+\cos^2\alpha) \implies \sin^2\alpha=\cos^2\alpha$$
and therefore $\alpha=45$ degrees.
\subsubsection{Drag}
Air resistance leads to a force acting on a projectile opposing the movement
$$F=-\frac{1}{2}m\mu \|v\|v$$
where $\|v\|=\sqrt{v_x^2+v_y^2}$ and $v=v_x+v_y$. This leads to
$$\frac{d}{dt}x(t)=v_x(t),\quad \frac{d}{dt}v_x(t)=-\frac{1}{2}\mu \|v\|v_x$$
$$\frac{d}{dt}y(t)=v_y(t),\quad \frac{d}{dt}v_y(t)=-g-\frac{1}{2}\mu \|v\|v_y$$
with initial conditions $x(0)=x_0$, $y(0)=y_0$, $v(x,0)=v_{x,0}$, and $v(y,0)=v_{y,0}$. This ODE does not have a closed form solution. We therefore need to approximate solutions.
\subsubsection{Initial Value Problems}
\begin{definition}
    An \textbf{initial value problem} is the task to find $x:I\to \R^d$ such that
    $$\frac{d}{dt}x(t)=F(t,x(t)),\quad x(t_0)=x_0$$
    for given initial value $x_0\in \R^d$ and source $F:I\times \R^d\to \R^d$.
\end{definition}
\begin{theorem}[Cauchy-Peano]
    Let $F$ be a continuous function. Then, provided $I$ is sufficiently small, there exists a solution to the IVP.
\end{theorem}
Notice that this only guarantees existence and not uniqueness. For uniquness, we must have additional regularity on $F$.
\begin{definition}
    The function $F:I\times \R^d\to \R^d$ is said to be \textbf{uniformly Lipschitz} if there exists $L>0$ such that
    $$\|F(t,x)-F(t,y)\|\leq L\|x-y\|,\quad \forall\:x,y\in \R^d.$$
\end{definition}
\begin{theorem}[Picard-Lindelöf]
    If $F$ also satisfies a uniform Lipschitz condition, then the solution to the IVP is unique.
\end{theorem}
It is important to note the following:
\begin{itemize}
    \item The interval may be small.
    \begin{example}
    $\frac{d}{dt}x(t)=(x(t))^2$ such that $x(1)=1$ has solution $y=1/(2-t)$ so $I=(-\infty,2)$. We have \textbf{finite-time blowup}.
    \end{example}
    \item The results are strict.
    \begin{example}
    $\frac{d}{dt}x(t)=2\sqrt{x(t)}$ such that $x(0)=0$ has infinitely many solutions. Fix $c>0$ and set
    $$x(t)=\begin{cases}
        0&t\leq c\\
        (t-c)^2&t>c.
    \end{cases}$$
    \end{example}
    \item Even when a unique solution exists, it can be very hard to solve the system for an explicit representation.
    \begin{example}
        The Lorenz system is a "simple weather model". We want to find $x(t),y(t),z(t)$ solving
        $$\frac{dx}{dt}=\sigma(y-x),\quad \frac{dy}{dt}=x(\rho-z),\quad \frac{dz}{dt}=xy-\beta z$$
        with $\sigma,\beta,\rho$ given.
    \end{example}
\end{itemize}
\subsection{Lecture 2: Time-Stepping Methods}
\subsubsection{Euler's Method}
Assuming that a solution exists on $[t_0,T]$ to the problem
$$\frac{d}{dt}x(t)=F(t,x(t)),\quad x(t_0)=x_0,$$
let us try to numerically approximate it. We consider a discretization $t_k=t_0+\tau k$ for $k=0,1,\ldots,N$, where $\tau=\frac{T-t_0}{N}$. We want to find a discrete approximation $\{y^n\}_{n=0}^{N}$ of $x(t)$ with $y^n\approx x(t_n)$ for $n=0,1,\ldots,N$. The idea is to approximate the differentiable operator by a difference operator:
$$\frac{d}{dt}x(t_n)\approx \frac{x(t_{n+1})-x(t_n)}{t_{n+1}-t_n}$$
which implies
$$\frac{y^{n+1}-y^n}{\tau}=F(t_n,y^n).$$
\begin{definition}[Euler's Method]
    Construct a sequence of approximations $\{y^{n}\}_{n=0}^{N}$ as follows:
    $$y^{n+1}=y^{n}+\tau F(t_n,y^n)$$
    where $y^0=x_0$.
\end{definition}
This is an explicit time-marching procedure as the RHS only depends on $t_n$ and $y^n$. How good is this approximation?
\begin{definition}
    We define the truncation error
    $$\pi^n=\frac{x(t_{n+1})-x(t_n)}{\tau}-F(t_n,x(t_n)).$$
\end{definition}
By Taylor's Theorem,
$$x(t_{n+1})=x(t_n)+\frac{d}{dt}x(t_n)(t_{n+1}-t_n)+\frac{1}{2}\frac{d^2}{dt^2}x(\xi_n)(t_{n+1}-t_n)^2$$
for some $\xi_n\in (t_{n},t_{n+1})$. Using that $F(t_n,x(t_n))=\frac{d}{dt}x(t_n)$ and substituting into the scheme,
$$\pi^n=\frac{x(t_n)+\frac{d}{dx}x(t_n)\tau+\frac{1}{2}\frac{d^2}{dx^2}x(\xi_n)\tau^2-x(t_n)}{\tau}-\frac{d}{dt}x(t_n)=\frac{1}{2}\tau \frac{d^2}{dt^2}x(\xi_n).$$
This implies the following.
\begin{lemma}
    The truncation error for Euler's method is given by
    $$\max_{n}\|\pi^n\|\leq \frac{1}{2}\tau \max_{\xi \in I}\left\|\frac{d^2}{dt^2}x(\xi)\right\|.$$
\end{lemma}
This is a first order approximation.
\subsubsection{Consistency and Stability}
\begin{definition}
    A one-step method is \textbf{consistent} with order $k$ if
    $$\max_{n}\|\tau^n\|\leq C\tau^k$$
    for some $C>0$. A one-step method is \textbf{convergent} with order $k$ if for the error $e^n=x(t_n)-y^n$,
    $$\max_{n}\|e^n\|\leq c\tau^k$$
    for some $c>0$.
\end{definition}
\begin{lemma}[Gronwall]
    Suppose we have monotone sequences $\{w_n\},\{b_n\}$ where $b_n$ are increasing and a constant $a>0$ such that
    $$w_0\leq b_0,\quad w_{n+1}\leq a\sum_{j=0}^{n}w_j+b_{n+1},\:\:n\geq 0.$$
    Then, $w_{n+1}\leq \exp((n+1)a)b_{n+1}$ for $n\geq 0$.
    \begin{proof}
        Set $S_{n+1}=a\sum_{j=0}^{n}w_j+b_{n+1}$. We now show $S_{n+1}\leq \exp((n+1)a)b_{n+1}$. First, we have $S_0\leq b_0$. Assume that it holds for $n$, that is, $S_n\leq \exp(na)b_n$ and by assumption $w_n\leq S_n$. Then,
        $$S_{n+1}-S_n=aw_n+b_{n+1}-b_n$$
        $$\implies S_{n+1}\leq (1+a)S_n+b_{n+1}-b_n\leq (1+a)e^{na}b_n+b_{n+1}-b_n\leq e^{a}e^{na}b_n+e^{(n+1)a}(b_{n+1}-b_n)\leq e^{(n+1)a}b_{n+1}.$$
        By induction, the result holds for all $n$.
    \end{proof}
\end{lemma}
\begin{theorem}[Discrete Stability of Euler's Method]
    Let $F$ be a Lipschitz continuous function with Lipschitz constant $L$. Let $x(t)$ be a solution to $\frac{d}{dt}x(t)=F(t,x(t))$ and $\{y^n\}$ generated by Euler's method. Then,
    $$\max_{n}\|e^n\|\leq \exp(LT)T\max_{n}\|\pi^n\|.$$
    \begin{proof}
        From the definition of error,
        \begin{align*}
            e^{n+1}&=x(t_{n+1})-y^{n+1} \\
            &=x(t_n)+\tau F(t_n,x(t_n))+\tau \pi^n-y^{n+1}\\
            &=x(t_n)+\tau F(t_n,x(t_n))+\tau \pi^n-y^n-\tau F(t_n,x(t_n))\\
            &=e^n+\tau\{F(t_n,x(t_n))-F(t_n,y^n)\}+\tau \pi^n.
        \end{align*}
        Taking the norm and applying the Lipschitz condition
        $$\|e^{n+1}\|\leq \|e^n\|+\tau L\|e^n\|+\tau\|\pi^n\|=(1+\tau L)\|e^n\|+\tau\|\pi^n\|.$$
        Recurisively, we obtain
        $$\|e^{n+1}\|\leq \tau L\sum_{j=0}^{n}\|e^j\|+\tau \sum_{j=0}^{n}\|\pi^j\|.$$
        From the Grownall lemma with $a=\tau L$, $b_{n+1}=\tau \sum_{j=0}^{n}\|\pi^j\|$ and $w_n=\|e^n\|$, we obtain
        $$\|e^{n+1}\|\leq \exp((n+1)L\tau)\tau \sum_{j=0}^{n}\|\pi^j\|.$$
        From here, the result follows.
    \end{proof}
\end{theorem}
An important principle is that consistency and stability imply convergence.
\subsubsection{Explicit Runge-Kutta Methods}
It is often necessary to construct time-stepping schemes that are more than first order convergent. A huge class of such schemes fit into the framework of a Runge-Kutta method.
\begin{definition}
    A Runge-Kutta time stepping scheme is of the form
    $$y^{n+1}=y^n+\tau \sum_{j=1}^{R}b_jK_j$$
    where
    $$K_1=F(t_n,y^n),\quad K_{j}=F\left(t_n+\tau c_j,y^n+\tau\sum_{i=1}^{j-1}a_{ij}K_i\right),\:\:j\geq 2.$$
\end{definition}
Thus, we need to find the coefficients $\{a_{ij},b_j,c_j\}$ the make this scheme converge with our desired order. They are often organized in a Butcher tableau:
\begin{table}[h]
\centering
\begin{tabular}{c|cccc}
$c_1$ & $0$ &  &  &  \\
$c_2$ & $a_{21}$ & 0 &  &  \\
$\vdots$ & $\vdots$ & $\vdots$ & $\ddots$ & \\
$c_R$ & $a_{R1}$ & $a_{R2}$ & $\cdots$ & 0 \\
\hline
      & $b_1$    & $b_2$    & $\cdots$ & $b_R$
\end{tabular}
\end{table}
For consistency, we require $\sum_{j}b_j=1$.
\begin{example}
    For $R=1$, the only possible choice is $b_1=1$ and we have the \textbf{explicit forward Euler scheme}
    $$y^{n+1}=y^n+\tau F(t_n,y^n).$$
\end{example}
\begin{example}
    For $R=2$, we have several choices. A popular choice is \textbf{Heun's method}
    $$y^{n+1}=y^n+\frac{\tau}{2}\left\{F(t_n,y^n)+F(t_n+\tau,y^n+\tau F(t_n,y^n))\right\}.$$
    Alternatively, we have a second order Euler scheme
    $$y^{n+1}=y^n+\tau\left\{F(t_n,y^n)+F(t_n+1/2\tau,y^n+1/2\tau F(t_n,y^n))\right\}.$$
\end{example}
\begin{example}
    For $R=4$, the classical \textbf{Runge-Kutta method} (RK4) is the fourth order scheme
    $$y^{n+1}=y^n+\frac{\tau}{6}(K_1+2K_2+2K_3+K_4),$$
    where
    $$K_1=F(t_n,y^n)$$
    $$K_2=F(t_n+1/2\tau,y^n+1/2\tau K_1)$$
    $$K_3=F(t_n+1/2\tau, y^n+1/2\tau K_2)$$
    $$K_4=F(t_n+\tau,y^n+\tau K_3).$$
\end{example}
\section{Day 2: Modeling}
\subsection{Lecture 1: Preliminaries}
\subsubsection{Integration by Parts}
We are all familiar with the integration by parts formula in $\R$:
$$\int_{a}^{b}uv'dx=(uv)\Big|_{a}^{b}-\int_{a}^b u'vdx.$$
To extend this we need the Divergence Theorem.
\begin{theorem}[Divergence Theorem]
Let $\Omega\subset \R^d$ be compact with smooth boundary $\partial \Omega$ and $x\in \R^d$. The exterior normal to $\Omega$ is denoted $n(x)$. Then, for any $F\in \mathcal{C}^1(V)$,
$$\int_{\Omega}\operatorname{div}F(x)dx=\int_{\partial \Omega}F(x)\cdot n(x)dS.$$
\end{theorem}
Using this, it is easy to obtain IBP in higher dimensions.
\begin{theorem}[Integration by Parts]
    Let $\Omega\subset \R^d$ be compact with smooth boundary $\partial \Omega$, $\phi:\Omega\to \R$, and $v:\Omega\to \R^d$. Then
    $$\int_{\Omega}v(x)\cdot \nabla \phi(x)dx=\int_{\partial \Omega}\phi(x)v(x)\cdot n(x)dS-\int_{\Omega}\phi(x)\operatorname{div}v\:dx.$$
    \begin{proof}
        Take $F(x)=\phi(x)v(x)$ in the Divergence Theorem. Then,
        $$\operatorname{div}(\phi v)=\phi\operatorname{div}(v)+v\cdot \nabla \phi.$$
        Thus,
        $$\int_{\Omega}v\cdot \nabla \phi=\int_{\Omega}\left[\operatorname{div}(\phi v)-\phi\operatorname{div}(v)\right]dx=\int_{\partial \Omega}\phi v\cdot n\:dS-\int_{\Omega}\phi \operatorname{div}(v)dx.$$
    \end{proof}
\end{theorem}
The most important identity to remember is
$$\int_{\Omega}v\partial_{x_i}u\:dx=\int_{\partial \Omega}uvn_i\:dS-\int_{\Omega}u\partial_{x_i}v\:dx.$$
From this, many formulas follow.
\begin{theorem}[Green's First Identity]
    $$\int_{\Omega}\nabla u\cdot \nabla v\:dx=\int_{\partial \Omega}v\nabla u\cdot n\:dS-\int_{\Omega}v\Delta u\:dx$$
    where $\Delta u=\operatorname{div}(\nabla u)$ is the Laplacian.
\end{theorem}
How do we use the formula in practice?
\begin{example}[2D Integration By Parts]
    \begin{align*}
        \int_{\Omega}\operatorname{div}(v)\phi&=\int_{\Omega}\frac{\partial v_1}{\partial x_1}\phi+\int_{\Omega}\frac{\partial v_2}{\partial v_2}\\
            &=\int_{\partial\Omega}\phi v_1n_1-\int_{\Omega}v_1\frac{\partial \phi}{\partial x_1}+\int_{\partial\Omega}\phi v_2n_2-\int_{\Omega}v_2\frac{\partial \phi}{\partial x_2} \\
            &=\int_{\partial\Omega}\phi v\cdot n-\int_{\Omega}v\cdot \nabla\phi.
    \end{align*}
\end{example}
\begin{example}[Non-Obvious Formula]
    $$\int_{\Omega}\nabla \times u\:dx=-\int_{\partial \Omega}u\times n\:ds.$$
    \begin{proof}
        Consider that
        $$\nabla \times u=\begin{bmatrix}
            \partial_{y}u_3-\partial_{z}u_2 \\
            \partial_{z}u_1-\partial_{x}u_3 \\
            \partial_{x}u_2-\partial_{y}u_1
        \end{bmatrix}.$$
        We can do integration by parts on each row and the formula follows.
    \end{proof}
\end{example}
\begin{example}[Another Non-Obvious Formula]
    $$\int_{\Omega}(\nabla \times u)\cdot v\:dx=\int_{\partial \Omega}(u\times v)\cdot n\:dS+\int_{\Omega}u\cdot(\nabla \times v)dx.$$
\end{example}
\subsubsection{Total Derivative}
Consider the function $f(t,x(t))$. Such a function could represent a quantity being convected/transpoted by a velocity $v(x,t)$. That is, $f(x(t),t)$ is convected by the velocity $v(x)=\frac{dx}{dt}$. Therefore, what is the rate of change of $f$? Using the chain rule,
$$\frac{d}{dt}f(x(t),t)=\frac{\partial f}{\partial x}\frac{\partial x}{\partial t}+\frac{\partial f}{\partial t}$$
or using $v$,
$$\frac{df}{dt}=\frac{\partial f}{\partial t}+v\frac{\partial f}{\partial x}.$$
We refer to this as the \textbf{total derivative} or \textbf{material derivative} of $f$. It denotes the rate of change of a quantity that is subjected to both a position and time dependent velocity field. In the multi-dimensional case, we can similarly obtain
$$\frac{d}{dt}f(x(t),t)=\frac{\partial f}{\partial t}+\sum_{i=1}^{d}\frac{\partial f}{\partial x_i}\frac{\partial x_i}{t}=\frac{\partial f}{\partial t}+\nabla_x f\cdot v.$$
More generally, this quantity is often called the \textbf{Lie derivative of 0-forms}.
\subsubsection{Integrals in Time-Dependent Domains}
Recall that for continuous $f:[a,b]\to \R$,
$$\int_{a}^{b}f(x)dx=F(b)-F(a),$$
that is, $f$ admits a primative $F$. If $a,b$ are time-dependent, i.e., we now are on the interval $[a(t),b(t)]$, then
$$\int_{a(t)}^{b(t)}f(s)ds=F(b(t))-F(a(t)).$$
Using the chain rule,
$$\frac{d}{dt}\left[\int_{a(t)}^{b(t)}f(s)ds\right]=f(b(t))b'(t)-f(a(t))a'(t).$$
This is often called the \textbf{Leibniz Rule} where the quantities $a'(t),b'(t)$ are \textbf{boundary velocities}. We can generalize this to the multi-dimensional setting, though we do not prove it.
\begin{theorem}[Reynold's Transport Theorem]
    Let $\Omega(t)$ be the domain of integration. Let $f=f(x,t)$ be scalar, vector, or tensor-valued. Then,
    $$\frac{d}{dt}\int_{\Omega(t)}f\:dV=\int_{\Omega(t)}\frac{\partial f}{\partial t}dV+\int_{\partial \Omega(t)}(v_b\cdot n)f\:dA$$
    where $n(x,t)$ is the outward pointing normal and $v_b$ is the velocity of the area element.
\end{theorem}
We could also alternative use the Divergence Theorem to reformulate Reynold's Transport Theorem as
$$\frac{d}{dt}\int_{\Omega(t)}fdx=\int_{\Omega(t)}(\frac{\partial f}{\partial t}+\operatorname{div}(vf))dx.$$
The quantity
$$\frac{\partial f}{\partial t}+\operatorname{div}(vf)$$
is referred to as the \textbf{Lie derivative of k-forms}.

\subsection{Lecture 2: Conservation Laws}
\subsubsection{Conservation of Mass}
Let $\Omega\subset \R^d$ with boundary $\partial \Omega$ be fixed. Let $\rho(x,t)$ denote the density (mass per unit volume) and $v(x,t)$ denote the velocity at whcih the mass moves. The total mass $M$ in $\Omega$ is given by
$$M=\int_{\Omega}\rho(x,t)dx.$$
Observe that mass can only enter or leave the boundary through the boundary so
$$\frac{d}{dt}\int_{\Omega}\rho(x,t)dx=-\int_{\partial \Omega}\rho(x,t)v(x,t)\cdot n\:dS.$$
Why do we negate the left side? If $v\cdot n>0$, then we have an outflow. If $v\cdot n<0$, then we have an inflow. Reorganizing, we obtain
$$\int_{\Omega}\frac{\partial \rho}{\partial t}+\operatorname{div}(\rho v)dx=0.$$
Since this holds for arbitrary $\Omega$, the integrand must be zero and
$$\frac{\partial \rho}{\partial t}+\operatorname{div}(\rho v)=0.$$
This is known as \textbf{conservation or balance of mass} as the inflow and outflow are balanced. $\rho$ is often called a \textbf{conserved quantity} if $v\cdot n=0$ on $\partial \Omega$. Indeed, then
$$0=\int_{\Omega}\frac{\partial \rho}{\partial t}+\operatorname{div}(\rho v)dx=\int_{\Omega}\frac{\partial \rho}{\partial t}dx+\int_{\partial \Omega}\rho v\cdot n\:dS=\int_{\Omega}\frac{\partial \rho}{\partial t}dx$$
$$\implies \frac{\partial}{\partial t}\int_{\Omega}\rho\:dx=0.$$
Thus,
$$\int_{\Omega}\rho(x,t)dx=\int_{\Omega}\rho(x,0)dx.$$
This means that the total mass at time $t$ is equivalent to the total mass at the starting time $t=0$.

\subsubsection{Conservation of Momentum}
For a force $F$ acting on a point mass, Newton's law says that
$$\frac{d}{dt}(mv)=F$$
where $mv$ is the momentum. If $\partial_{t}m=0$, then we can rewrite this as
$$m\frac{dv}{dt}=F.$$
$v=v(x(t),t)$ so we interpret the acceleration $\frac{dv}{dt}$ as a total derivative. Since $\rho$ is the mass per unit volume,
$$\rho\left(\frac{\partial v}{\partial t}+\nabla v v\right)=F.$$
But $v$ is a vector, so we interpret
$$[\nabla v]_{ij}=\frac{\partial v_i}{\partial x_j}$$
as the Jacobian of $v$. More commonly, we rewrite
$$\rho\left(\frac{\partial v}{\partial t}+(v\cdot \nabla)v\right)=F.$$
The right hand side are the forces per unit volume. What types of forces act on bodies? There are two primary types.
\begin{itemize}
    \item External forces. Act on the exterior of the body.
    \item Cohesive or internal forces: Generated by the interior of the body.
\end{itemize}
\begin{example}[Gravity]
    Gravity is an external force
    $$F_{g}=\begin{bmatrix}
        0 \\
        0 \\
        -g
    \end{bmatrix}$$
    where $g\approx 9.81 m/s^2$ is the graviational constant.
\end{example}
Internal forces are mathematically represented by tensors, that is, $F=\operatorname{div}(\sigma)$ where $\sigma\in \R^{d\times d}$. The divergence of a matrix is defined as
$$(\operatorname{div}\sigma)_{i}=\sum_{j=1}^{d}\frac{\partial}{\partial x_j}(\sigma_{ij}).$$
$\sigma$ is called the \textbf{stress tensor} and characterizes the elastic/compressible nature of the substance. The precise formula of $\sigma$ depends on the material or quantity of interest. It can be quite complicated, but we assume that we are given a precise formula $\sigma:\Omega\to \R^{n\times n}$.
\begin{example}[Pressure]
    Consider $\Omega=\Omega_1\cup \Omega_2$ with the interface (boundary between subdomains) $\Gamma$. Let $n$ be the normal to $\Gamma$. We can define the \textbf{traction vector} $T=\sigma n\big|_{\Gamma}$. This characterizes the force per unit area resulting from cohesive stress. Let us define the \textbf{pressure} $p=-\frac{1}{3}(\sigma_{11}+\sigma_{22}+\sigma_{33})=-\frac{1}{3}\operatorname{tr}(\sigma)$. The simplest form of the stress tensor (commonly seen in fluids) is
    $$\sigma=-pI=\begin{bmatrix}
        -p & 0 & 0 \\
        0 & -p & 0 \\
        0 & 0 & -p
    \end{bmatrix}.$$
    In this context, $T=\sigma n=-pn$. If $p>0$, $T$ points opposite to $n$ and we refer to this as \textbf{compression}. If $p<0$, then $T$ points in the same direction as $n$ and we refer to this as an \textbf{attraction state}. Fluids and gasses can withstand compression but they do not support traction. In contrast, solids withstand both compression and traction.
\end{example}
In general therefore, we may write $F=F_{e}+\operatorname{div}(\sigma)$ where $F_e$ denotes the external force. So far, we have conservation of mass and momentum:
$$\begin{cases}
    \frac{\partial \rho}{\partial t}+\operatorname{div}(\rho v)=0, \\
    \rho\left[\frac{\partial v}{\partial t}+(v\cdot \nabla)v\right]-\operatorname{div}(\sigma)=F_e.
\end{cases}$$
Note that conservation of momentum is not written in divergence form. To fix this, we multiply the mass law by $v$ and add them:
$$\frac{\partial \rho}{\partial t}v+\operatorname{div}(\rho v)v=0$$
$$\implies \frac{\partial \rho}{\partial t}v+\rho\frac{\partial v}{\partial t}+\operatorname{div}(\rho v)v+\rho (v\cdot \nabla)v-\operatorname{div}(\sigma)=F_e$$
$$\implies \frac{\partial}{\partial t}(\rho v)+\operatorname{div}(\rho vv^T)-\operatorname{div}(\sigma)=F_e.$$
Here, we used that
$$\operatorname{div}(\rho vv^T)=\sum_{j=1}^{d}\partial x_{j}(\rho v_iv_j)=\sum_{j=1}^{d}v_i\underbrace{\partial_{x_j}(\rho v_j)}_{\operatorname{div}(\rho v)}+\rho v_j \underbrace{\partial_{x_j}v_i}_{\nabla v}=v\operatorname{div}(\rho v)+\rho \nabla v v=\operatorname{div}(\rho v)v+\rho(v\cdot \nabla)v.$$
Thus in divergence form, the conservation of mass and momentum is
$$\begin{cases}
    \frac{\partial \rho}{\partial t}+\operatorname{div}(\rho v)=0, \\
    \frac{\partial}{\partial t}(\rho v)+\operatorname{div}(\rho vv^T-\sigma)=F_e.
\end{cases}$$

\subsubsection{Conservation of Energy}
This system is still incomplete, however. Consider the \textbf{power} of the forces applied to the system, $P=P(\rho,e)$. Here, $e$ is the \textbf{specific internal energy} and is related to the \textbf{temperature} $\theta$. This system does not describe the evolution of $e$. We assume that the gas/fluid of interest is purely theorem-mechanical. A thermo-mechanical body of fluid can only obtain energy in the form of internal and kinetic energy. In this case, the mechanical energy density
$$\varepsilon=\underbrace{\rho e}_{\text{internal}}+\underbrace{\frac{1}{2}\rho |v|^2}_{\text{kinetic}}.$$
We define
$$E=\int_{\Omega}\varepsilon\:dx$$
as the total thermo-mechanical energy stored in $\Omega$. The \textbf{First Law of Thermodynamics} says that
$$\frac{dE}{dt}=\frac{d}{dt}\int_{\Omega(t)}\varepsilon\:dx=P+Q_s$$
where the rate of heat recieved by the system is
$$Q_s=\int_{\Omega(t)}\Gamma\:dx+\int_{\partial\Omega(t)}q\cdot n\:dS.$$
Here, $\Gamma$ is the \textbf{source of the heat} in the bulk of $\Omega$, $q$ is the \textbf{flux} of the heat through $\partial \Omega$.
The aformentioned power of the system is
$$P=\int_{\Omega(t)}F_e\cdot v\:dx+\int_{\partial \Omega(t)}(\sigma n)v\:dS.$$
Therefore,
$$\frac{d}{dt}\int_{\Omega(t)}\varepsilon\:dx=\int_{\Omega(t)}F_e\cdot v+\Gamma\:dx+\int_{\partial\Omega(t)}(\sigma n)\cdot v+qn\:dS.$$
Using the Reynold Transport Theorem and the Divergence Theorem,
$$\int_{\Omega(t)}\frac{\partial \varepsilon}{\partial t}+\operatorname{div}(\varepsilon v)-\operatorname{div}(\sigma v)-\operatorname{div}(q)\:dx=\int_{\Omega(t)}F_e\cdot v+\Gamma\:dx.$$
Since this holds true for any $\Omega(t)$,
$$\frac{\partial \varepsilon}{\partial t}+\operatorname{div}(\varepsilon v-\sigma v-q)=F_e\cdot v+\Gamma.$$
This is \textbf{balance of total mechanical energy}. Note that if there are no external forces, $F_e=0$, and if there are no sources of heat, $\Gamma=0$. So \textbf{conservation of total mechanical energy} is
$$\frac{\partial \varepsilon}{\partial t}+\operatorname{div}(\varepsilon v-\sigma v-q)=0.$$
Now let's consider a fixed domain, i.e., $\frac{\partial \Omega}{\partial t}=0$. If $v\cdot n|_{\partial \Omega}=0$, $v^T\sigma n|_{\partial \Omega}=0$, $q\cdot n|_{\partial \Omega}=0$, then
$$\int_{\Omega}\frac{\partial \varepsilon}{\partial t}+\operatorname{div}(\varepsilon v-\sigma v-q)dx=0$$
$$\implies \int_{\Omega}\frac{\partial \varepsilon}{\partial t}dx=-\int_{\Omega}(\varepsilon v\cdot n-(\sigma v)\cdot n-q\cdot n)dS=0$$
$$\implies \int_{\Omega}\epsilon(t)dx=\int_[\Omega]\epsilon(0)dx$$
and so total mechanical energy is a conserved quantity.

So far, we derived an evolution equation for $\varepsilon$. By definition
$$\varepsilon=\rho e+\frac{1}{2}\rho\|v\|^2 \implies \rho e=\varepsilon-\frac{1}{2}\rho\|v\|^2$$
is the internal energy. Let's derive an evolution equation for $\rho e$. We have that
$$\frac{\partial (\rho e)}{\partial t}=\frac{\partial \varepsilon}{\partial t}-\frac{\partial}{\partial t}(1/2\rho|v|^2).$$
The first term we know is
$$\frac{\partial \varepsilon}{\partial t}=-\operatorname{div}(\varepsilon v-\sigma v-q)+F_e\cdot v+\Gamma.$$
For the second term,
$$\frac{\partial}{\partial t}(1/2\rho |v|^2)=\frac{1}{2}\partial_{t}\rho |v|^2+\rho v\partial_{t}v.$$
For the first part of this, we multiply conservation of mass by $\frac{1}{2}|v|^2$ so that
$$\partial_{t}\rho \frac{1}{2}|v|^2+\operatorname{div}(\rho v)\frac{1}{2}|v|^2=0.$$
For the second part of this, we multiply conservation of momentum by $v$,
$$\rho \frac{\partial v}{\partial t}\cdot v+(\nabla vv)\cdot v-\operatorname{div}(\sigma)\cdot v=F_e\cdot v.$$
Adding both, we get an evolution of kinetic energy:
$$\frac{\partial }{\partial t}\left(\frac{1}{2}\rho|v|^2\right)+\operatorname{div}\left(\frac{1}{2}|v|^2v\right)-\operatorname{div}(\sigma v)=F_e\cdot v.$$
Now we can compute $\partial_{t}(\rho e)$. We have that
$$\frac{\partial}{\partial t}(\rho e)=\frac{\partial \varepsilon}{\partial t}-\frac{\partial}{\partial t}\left(\frac{1}{2}\rho |v|^2\right)=-\operatorname{div}(\rho e v)-\operatorname{div}(q)+\underbrace{\div(\sigma v)-\operatorname{div}(\sigma)}_{=\sigma:\nabla v}\cdot v+\Gamma.$$
Here, we use the double contraction
$$\sigma:\nabla v=\sum_{i=1}^{d}\sum_{j=1}^{d}\sigma_{ij}\frac{\partial v_i}{\partial x_j}.$$
Reorganizing, we express \textbf{balance of internal energy} as
$$\frac{\partial (\rho e)}{\partial t}+\operatorname{div}(\rho ev+q)=\sigma:\nabla v+\Gamma.$$
Using similar arguments we can deduce that
$$\rho\left[\frac{\partial e}{\partial t}+v\cdot \nabla e\right]+\operatorname{div}(q)=\sigma:\nabla v+\Gamma.$$

\subsection{Lecture 3: Physical Models}
\subsubsection{Euler Equations}
Assume that we have no heat sources or heat conduction so that $\Gamma=0$, $q=0$. Also assume that we have a diagonal stress tensor $\sigma=-pI$. Then, we obtain the system of conservation laws
$$\begin{cases}
    \frac{\partial \rho}{\partial t}+\operatorname{div}(\rho v)=0 \\
    \frac{\partial (\rho v)}{\partial t}+\operatorname{div}(\rho vv^T+Ip)=0 \\
    \frac{\partial \varepsilon}{\partial t}+\operatorname{div}[(\varepsilon+p)v]=0.
\end{cases}$$
We still have to objectify the pressure $p$. The simplest case is for an ideal gas.
$$p=(\gamma-1)\rho e,\quad \rho e=\varepsilon-\frac{1}{2}\rho|v|^2,$$
with temperature $\theta=(\gamma+1)e$ and $1<\gamma<5/3$ the ratio of specific heat. This is known as a \textbf{thermo-mechanical closure}. This system is of somewhat universal validity. Indeed, the formula for $p$ is phenomonological and depends on the gas we model. A closure is a "constitutive relationship". Through this formulation, we obtain \textbf{Euler's Equations for Gas Dynamics}.

\subsubsection{Acoustic Wave Equation}
We could also consider the simplified system
$$\begin{cases}
    \frac{\partial \rho}{\partial t}+\operatorname{div}(\rho v)=0 \\
    \frac{\partial (\rho v)}{\partial t}+\nabla p=0.
\end{cases}.$$
Differentiating the first equation and taking the divergence of the second,
$$\begin{cases}
    \frac{\partial^2 \rho}{\partial t^2}+\frac{\partial}{\partial t}\operatorname{div}(\rho v)=0 \\
    \frac{\partial}{\partial t}\operatorname{div}(\rho v)=-\Delta \rho.
\end{cases}$$
Combining, we have
$$\frac{\partial^2\rho}{\partial t}-\Delta P=0.$$
With the \textbf{isothermal closure} $p=c^2\rho$, we obtain the \textbf{acoustic wave equation}
$$\frac{\partial^2\rho}{\partial t}-c^2\Delta \rho=0.$$
In the 1D case, this admits a solution as a travelling wave
$$\rho(x,t)=f(ct-x)+g(ct+x)$$
for any functions $f,g$.

\subsubsection{Advection-Diffusion-Reaction}
Finally, consider
$$\rho \frac{\partial e}{\partial t}+\rho v\cdot \nabla e+\operatorname{div}(q)=\sigma:\nabla v+\Gamma.$$
For gases, $\theta=(\gamma-1)e$ so internal energy corresponds with temperature. For incompressible substances like solids, $e=c_p\theta$, $\rho\rho_0$ is constant (incompressible flow), and the power of viscous stress $\sigma:\nabla v$ is neglibile. Then, we get
$$\frac{\partial \theta}{\partial t}+v\cdot \nabla \theta+\frac{1}{c_p\rho_0}\operatorname{div}(q)=\frac{\Gamma}{c_p\rho_0}.$$
When $q=-k\nabla \theta$,
$$\frac{\partial \theta}{\partial t}+\underbrace{v\cdot \nabla \theta}_{\text{advection}}-\underbrace{\frac{1}{c_p\rho_0}\Delta \theta}_{\text{diffusion}}=\underbrace{\frac{\Gamma}{c_p\rho_0}}_{\text{reaction}}.$$
This is the \textbf{advection-diffusion-reaction equation}. When $v=0,\Gamma=0$, we obtain the \textbf{heat/diffusion equation}
$$\frac{\partial \theta}{\partial t}-\frac{k}{c_p\rho_0}\Delta \theta$$
and the quantity $k/c_p\rho_0$ is the \textbf{thermal diffusivity}.

\section{Day 3: Partial Differential Equations}
\subsection{Lecture 1: Parabolic Equations}
\subsubsection{Problem Statement}
The goal of this lecture is to investigate the structure of the following scalar-valued initial boundary value problem. We want to find $u:\R\times \R^+\to \R$ such that
\begin{align*}
    \partial_{t}u-p\partial_{xx}u+q\partial_{x}u+ru&=f,\quad x\in \R,t\in \R^+ \\
    u(x,0)&=u_0(x) \\
    \lim_{x\to \pm \infty}u(x,t)&=0, \quad \forall\:t\in \R^+.
\end{align*}
Here $p,q,r\in \R$ are constants such that $p>0$ and $\geq 0$, and $f(\cdot ,t),u_0\in \mathbb{L}^2(\R)$. We say that $f$ is the \textbf{source} and $u_0$ is the \textbf{initial data}. The conditions $\lim_{x\to \pm \infty}u(x,t)$ are called Dirichlet boundary conditions at infinity and enforce suitable decay in our function. More generally, one could also consider a finite interval $(-L,L)$ and prescribe that $u(\pm L,t)=0$.
\begin{remark}
    If we replace $\partial t$ by $t$, $\partial_{x}$ by $x$, and $\partial_{xx}$ by $x^2$, then we would obtain $(t-px^2+qx+r)u=f$ (the reason for this is motivated by Fourier analysis). Since we assumed $p>0$, the two-variate polynomial generates a \textbf{parabola} and hence it is customary to say that this problem is \textbf{parabolic}.
\end{remark}
The first three natural questions that come to mind are
\begin{enumerate}
    \item \textbf{Existence}: Does there exist a solution?
    \item \textbf{Uniqueness}: If a solution exists, is it unique?
    \item \textbf{Stability}: If a solution exists, is the solution operator $(f,u_0)\mapsto f$ continuous in some sense? In other words, if $\|(f^{(1)},u_0^{(1)})-(f^{(2)},u_0^{(2)})\|<\delta$, is $\|u^{(1)}-u^{(2)}\|<\epsilon$ in some suitable metric $\|\cdot\|$?
\end{enumerate}
It turns out the existence question is the most difficult one as it requires significant analysis. The easiest question is that concerning the stability/continuity. Hence, we are go-
ing investigate first, the stability, then the uniqueness, and finally say a few words
regarding existence.
\subsubsection{Stability}
The \textbf{energy method} is a principle that we will use to derive a-priori estimates for PDEs. The idea is to multiply the PDE by the solution itself and integrate by parts to obtain a useful bound. Suppose that such a solution $u$ to our PDE exists.
\begin{theorem}
    Let $u(x,t)$ be a solution and $T>0$. Then,
    $$\|u(\cdot, T)\|_{L^2(\R)}\leq \|u_0\|_{L^2(\R)}+\int_{0}^{T}\|f(\cdot,t)\|_{L^2(\R)}dt.$$
    \begin{proof}
        Using the energy method, we multiply the PDE by $u$ and integrate:
        $$\int_{\R}u\partial_{t}u-pu\partial_{xx}u+qu\partial_{x}u+ru^2=\int_{\R}uf.$$
        We have that,
        \begin{align*}
            \int_{\R}\left(u\partial_{t}u-pu\partial_{xx}u+qu\partial_{x}u+ru^2\right)&=\int_{R}\left(\partial_{t}\left(\frac{1}{2}u^2\right)-pu\partial_{xx}u+q\partial_{x}\left(\frac{1}{2}u^2\right)+ru^2\right)\\
            &=\partial_{t}\int_{\R}\frac{1}{2}u^2+\int_{\R}-pu\partial_{xx}u+\int_{\R}\left(q\partial_{x}\left(\frac{1}{2}u^2\right)+ru^2\right) \\
            &=\partial_{t}\int_{\R}\frac{1}{2}u^2-\underbrace{(\lim_{N,N'\to \infty}u(x,t)\partial_{x}u(x,t)\Big|_{-N'}^{N})}_{0}+\int_{\R}p(\partial_{x}u)^2+\int_{\R}\left(q\partial_{x}\left(\frac{1}{2}u^2\right)+ru^2\right) \\
            &=\partial_{t}\int_{\R}\frac{1}{2}u^2+\int_{\R}p(\partial_{x}u)^2+\int_{\R}q\partial_{x}\left(\frac{1}{2}u^2\right)+\int_{\R}ru^2 \\
           &=\partial_{t}\int_{\R}\frac{1}{2}u^2+\int_{\R}p(\partial_{x}u)^2+\underbrace{\lim_{N,N'\to \infty}\left(\frac{1}{2}u^2(N,t)-\frac{1}{2}u^2(N',t)\right)}_{0}+\int_{\R}ru^2 \\
           &=\frac{1}{2}\partial_{t}\|u(\cdot,t)\|^2_{L^2(\R)}+p\|\partial_{x}u\|^2+r\|u(\cdot,t)\|^2.
        \end{align*}
        Thus, this together with the Cauchy-Schwarz inequality implies
        $$\frac{1}{2}\partial_{t}\|u(\cdot,t)\|^2_{L^2(\R)}+r\|u(\cdot,t)\|^2\leq \int_{\R}uf\leq \|u(\cdot,t)\|_{L^2(\R)}\|f(\cdot,t)\|_{L^2(\R)}.$$
        By noting that $\frac{1}{2}\partial_{t}\|u(\cdot,t)\|^2_{L^2(\R)}=\|u(\cdot,t)\|_{L^2(\R)}\partial_{t}\|u(\cdot,t)\|_{L^2(\R)}$ (by Leibniz rule), we infer that
        $$\partial_{t}\|u(\cdot,t)\|_{L^2(\R)}+r\|u(\cdot,t)\|_{L^2(\R)}\leq \|f(\cdot,\R)\|_{L^2(\R)}.$$
        Dropping the term $r\|u(\cdot,t)\|_{L^2(\R)}$ and integrating in time, this gives the resulting $L^2$ a-priori error estimate
        $$\|u(\cdot,T)\|_{L^2(\R)}\leq \|u_0\|_{L^2(\R)}+\int_{0}^{T}\|f(\cdot,t)\|_{L^2(\R)}dt.$$
    \end{proof}
\end{theorem}
\begin{theorem}
    Let $u(x,t)$ be a solution and $T>0$. Then, we have the refined a-priori estimate
    $$\|u(\cdot,T)\|_{L^2(\R)}\leq e^{-rT}\|u_0\|_{L^2(\R)}+\int_{0}^{T}e^{r(t-T)}\|f(\cdot,t)\|_{L^2(\R)}dt.$$
    \begin{proof}
        The above computations give
        $$\partial_{t}(e^{rt}\|u(\cdot,t)\|_{L^2(\R)})=e^{rt}\partial_{t}\|u(\cdot,t)\|_{L^2(\R)}+re^{rt}\|u(\cdot,t)\|_{L^2(\R)}\leq e^{rt}\|f(\cdot,t)\|.$$
        Integrating in time, we have the desired result.
    \end{proof}
\end{theorem}
\begin{theorem}
    Let $u(x,t)$ be a solution and $T>0$. Then, we have the a-priori estimate
    $$\frac{1}{2}\|u(\cdot,t)\|^2_{L^2(\R)}+\int_{0}^{T}\left(p\|\partial_{x}u(\cdot,t)\|^2_{L^2(\R)}+r\|u(\cdot,t)\|^2_{L^2(\R)}\right)dt\leq \frac{1}{2}\|u_0\|^2_{L^2(\R)}+\frac{1}{2r}\int_{0}^{T}\|f(\cdot,t)\|^2_{L^2(\R)}dt.$$
    \begin{proof}
        By Young's inequality, we obtain that
        $$\int_{\R}uf\leq \|u(\cdot,t)\|_{L^2(\R)}\|f(\cdot,t)\|_{L^2(\R)}\leq \frac{r}{2}\|u(\cdot,t)\|^2_{L^2(\R)}+\frac{1}{2r}\|f(\cdot,t)\|^2_{L^2(\R)}.$$
        Then we obtain that
        $$\frac{1}{2}\partial_{t}\|u(\cdot,t)\|^2_{L^2(\R)}+p\|\partial_{x}u(\cdot,t)\|^2_{L^2(\R)}+r\|u(\cdot,t)\|^2_{L^2(\R)}\leq \frac{r}{2}\|u(\cdot,t)\|^2_{L^2(\R)}+\frac{1}{2r}\|f(\cdot,t)\|^2_{L^2(\R)}.$$
        It follows that
        $$\frac{1}{2}\partial_{t}\|u(\cdot,t)\|^2_{L^2(\R)}+p\|\partial_{x}u(\cdot,t)\|^2_{L^2(\R)}+r\|u(\cdot,t)\|^2_{L^2(\R)}\leq \frac{1}{2r}\|f(\cdot,t)\|^2_{L^2(\R)}.$$
        Integrating in time, we obtain the desired result.
    \end{proof}
\end{theorem}
We are now ready to obtain stability using these error estimates.
\begin{theorem}
    Let $u^1$ and $u^2$ correspond to the data $(f^1,u^1_0),(f^2,u^2_0)$, respectively. Then,
    $$\|(u^1-u^2)(\cdot,T)\|_{L^2(\R)}\leq \|u^1_0-u^2_0\|_{L^2(\R)}+\int_{0}^{T}\|(f^1-f^2)(\cdot,t)\|_{L^2(\R)}dt.$$
    \begin{proof}
        Let $\phi=u^1-u^2$. Then linearity implies that
        \begin{align*}
            \partial_{t}\phi-p\partial_{xx}\phi+r\phi&=f^1-f^2,\quad x\in\R,t\in \R^+ \\
            \phi(x,0)&=u_0^1(x)-u_0^2(x) \\
            \lim_{x\to \pm \infty}\phi(x,t)&=0.
        \end{align*}
    \end{proof}
    We can apply the first a-priori estimate to this and obtain the stability bound.
\end{theorem}
We have discovered a notion of continuity for the solution operator. If the difference in data is small, then the difference in the corresponding solutions will be small as well. Moreover, if we consider a sequence $\{f^n,u^n_0\}$ such that $f^n\to f$ and $u^n_0\to u_0$ in $L^2(\R)$, then $u^n\to u$ in $L^2(\R)$. A similar argument holds by using the second estimate. Many more classes of stability bounds can be obtained by using other a-priori estimates.
\begin{remark}
    The last estimate shows that a good candidate for a smoothness class where the existence of a solution could be established is the space composed of the functions $v:\R\times \R^+\to \R$ for which $t\mapsto \|v(\cdot,t)\|_{L^2(\R)}$ is continuous and the following quantity is bounded:
    $$\int_{0}^{T}(p\|\partial_{x}u\|^2_{L^2(\R)}+r\|u\|^2_{L^2(\R)})dt<\infty$$
    for all $T>0$. We define the space
    $$X=\left\{v:\R\times \R^+\to \R\:|\:v(\cdot,t)_{L^2(\R)}\in C(\R^+,\R),\:\int_{0}^{T}(p\|\partial_{x}u\|^2_{L^2(\R)}+r\|u\|^2_{L^2(\R)})dt<\infty,\:\forall\:T>0\right\}.$$
    The estimate also shows that it is likely to construct a sufficient condition on the data $(f,u_0)$ for the existence of a solution where $\|u_0\|_{L^2(\R)}$ is bounded and $\int_{0}^{T}\|f(\cdot,t)\|_{L^2(\R)}dt$ are bounded for all $T>0$. Accordingly, we define,
    $$Y=\left\{(f,u_0)\:|\:\|u_0\|_{L^2(\R)}<\infty,\:\:\int_{0}^{T}\|f(\cdot,t)\|_{L^2(\R)}\|dt<\infty,\:\:\forall\:T>0\right\}.$$
   Here, we equip $X$ and $Y$ with the natural norm and corresponding topology.
\end{remark}
\subsubsection{Uniqueness}
Using stability, it is easy to obtain uniqueness.
\begin{theorem}
    Consider $u^1$ and $u^2$ as solutions to the PDE in the space $X$ for the same data $(f,u_0)$ in the normed space $Y$. Then, $u^1=u^2$.
    \begin{proof}
        The stability bound implies that for any $T>0$,
        $$\|(u^1-u^2)(\cdot,T)\|_{L^2(\R)}\leq \underbrace{\|u_0-u_0\|_{L^2(\R)}}_{0}+\int_{0}^{T}\underbrace{\|(f-f)(\cdot,t)\|_{L^2(\R)}}_{0}dt=0$$
        and therefore $\|(u^1-u^2)(\cdot,T)\|_{L^2(\R)}=0$. But this implies $u^1(\cdot, T)=u^2(\cdot,T)$ for all $T$ and therefore $u^1=u^2$.
    \end{proof}
\end{theorem}
\subsubsection{Existence}
Proving existence of a solution in $X$ with data in $Y$ is quite technical. There are many methods of doing so. For example, one may approach the problem by using a Fourier transform in space. Another method that is closer to numerical analysis consists of constructing
finite-dimensional approximations that are uniformly bounded in $X$ and passing to the limit. This second method works particularly well when the space domain is
a bounded interval $(-L, L)$. A general existence result for parabolic equations is known in the literature as Lions' theorem.
\subsubsection{More on Boundary Conditions}
We assume in this section that the model problem is set over the finite interval
$(-L,L).$ In this case, many boundary conditions can be enforced. As we have seen
above, deriving a priori estimates is essential to define a smoothness class where one
can prove existence, uniqueness, and stability of a solution. All the arguments in-
voked above using the energy method can be applied. The key point is the integration by parts in
\begin{align*}
    -\int_{-L}^{L}pu\partial_{xx}u+\int_{-L}^{L}qu\partial_{x}u &=\int_{-L}^{L}p(\partial_{x}u)^2+\int_{-L}^{L}q\partial_{x}\frac{1}{2}u^2-pu\partial_{x}u\Big|_{-L}^{L} \\
    &=\int_{-L}^{L}p(\partial_{x}u)^2+(q\frac{1}{2}u^2-pu\partial_{x}u)\Big|_{-L}^{L}
\end{align*}
Then, admissible boundary conditions are obtained by ensuring that the boundary
terms appearing above produce non-negative terms. For instance, we could try to
enforce
$$u(L)\left(q\frac{1}{2}u(L)-p\partial_{x}u(L)\right)\geq 0$$
$$u(-L)\left(q\frac{1}{2}u(-L)-p\partial_{x}u(-L)\right)\leq 0.$$
This can be achieved by enforcing Dirichlet boundary conditions:
$u(L)=u(-L)=0$.
If $q \geq 0$, one can also enforce a Dirichlet boundary condition at $-L$ and a Neumann
boundary condition at $+L$: $u(-L)=0$, $\partial_{x}u(L)=0$ and the other way around if $q\leq 0$. One can also enforce Robin boundary conditions
$-\partial_{x}u(L)=Hu(L)$, and $\partial_{x}u(-L)=Hu(-L)$ where $H$ is such that $H>-\frac{1}{2}q$.e

\subsection{Lecture 2: Scalar Conservation Equations}
\subsubsection{Problem Statement}
The goal of this lecture is to investigate the following nonlinear partial differential equation. We want to find $u:\R\times \R^+\to \R$ so that
\begin{align*}
    \partial_t u+\partial_{x}(f(u))&=0,\quad x\in \R,x\in \R^+ \\
    u(x,0)&=u_0(x),\quad x\in \R \\
    \lim_{x\to \pm \infty}(u(x,t)-u_0(x))&=0.
\end{align*}
Here $f:\R\to \R$ is the \textbf{flux} and is a locally Lipschitz function, while $u_0$ is the initial data.
\begin{example}
    \begin{enumerate}
        \item Linear tranport: $f(v)=\beta v$, $\beta\neq 0$.
        \item Burgers equation: $f(v)=\frac{1}{2}v^2$.
        \item Traffic flow equations: $f(p)=u_{\rm max}\rho(1-\rho/\rho_{\rm max})$.
        \item Buckley-Leverett equation: $f(v)=\frac{v^2}{v^2+(1-v)^2}.$
    \end{enumerate}
\end{example}
We again consider the problem of well-posedness, that is, obtaining existence, uniqueness, and stability.
\subsubsection{Method of Characteristics}
Let us assume that we have a unique solution $u(x, t)$ and let us assume that this
solution is locally Lipschitz with respect to x and continuous with respect to t, at
least over some time interval $t\in (0,T)$
The idea is to introduce a change a variable
based on $u$.
\begin{definition}
    For $s\in \R$, the curve $\{(X(s,t),t)\:|\:t\geq 0\}$ in $\R\times [0,\infty)$ is called a \textbf{characteristic} where $X:\R\times \R^+\to \R$ solves
    \begin{align*}
        X_t(s,t)&=f'(u(X,st),t),\quad s\in \R,t>0 \\
        X(s,0)&=s,\quad s\in \R.
    \end{align*}
\end{definition}
\begin{remark}
    Notice that, owing to the assumption we made on the solution $u$, the Cauchy-Lipchitz
theorem (a.k.a. Picard-Lindelöf theorem) implies that $X(s, t)$ is well defined for all $s\in \R,t\in (0,T)$.
\end{remark}
For the time being the situation looks desperate since
$X(s, t)$ is defined by invoking $u$ which is still unknown, but a little miracle will
happen and will solve this conondrum. Consider a new functon $\phi:\R\times \R^+\to \R$ defined by $\phi(s,t)=u(X(s,t),t)$. Then, by the chain rule
\begin{align*}
    \phi+t(s,t)=u_{x}(X(s,t),t)\underbrace{X_t(s,t)}_{f'(u(X(s,t)))}+u_t(X(s,t),t) \\
    &=[u_xf'(u)+u_t](X(s,t),t)=0
\end{align*}
which implies that
$$\phi(s,t)=\phi(s,0)=u(X(s,0),0)=u_0(X(s,0))=u_0(s).$$
Moreover, since we have that $X_t(s,t)=f'(u_0(s))$, then solving the ODE we obtain the implicit representation
$$u(X(s,t),t)=u_0(s),\quad X(s,t)=s+tf'(u_0(s)).$$
Thus, the characteristics are really just straight lines. Given $s\in \R$ and $t>0$, the value of $u$ at the location $X(s,t)$ and time $t$ is $u_0(s)$.
Obtaining an explicit representation of the solution to (1) using the methods of
characteristics is in general nontrivial. This is done by expressing $s$ as a function of
$x$ and $t$. Let $x\in \R$, $t>0$. To find $u$, we must find $s\in \R$ such that
$$s+tf'(u_0(s))=x.$$
This equatuon is nonlinear from the presence of of $f'(u_0(s))$. We can apply the Implicit Function Theorem to $G(s)=0$ with $G(s)=s+tf'(u(s))-x$. Let $f\in C^2(\R)$. If $1+tf''(u_0)\partial_{x}u_0(x)\neq 0$, then there is an $S(x,t)$ such that $G(S(x,t))=0$. With this function $S(x,t)$, we have
$$u(x,t)=u_0(S(x,t)).$$
In general, we have the following theorem:
\begin{theorem}
    Assume that $f\in C^2(\R)$,$u_0\in C^1(\R)$, and $\inf_{\R}\{f''(u_0)u_0'\}>-\infty$ (essential lower bound). Then, the problem has a unique solution $u$ over $t\in (0,T^*)$, where
    $$T^*=\begin{cases}
        \infty&\inf\{f''(u_0)u_0'\}\geq 0 \\
        -\frac{1}{\inf\{f''(u_0)u'_0\}}&\{f''(u_0)u'_0\}<0.
    \end{cases}$$
\end{theorem}
Let us assume that $u_0\in C^1(\R)$. If $u_0,f$ are such that $1+tf''(u_0(s))\partial_{x}u_0(s)\neq 0$, for all $s\in \R$ and $t>0$, then $S(x,t)$ is always well-defined. In this case $T^*=\infty$. This above situation occurs when $f$ is convex and $u_0$ is montonically increasing. The same conclusion holds if $f$ is concave and $u_0$ is montonically decreasing.
\begin{example}
    Consider the transport equation where $f(v)=\beta v$ for $\beta \neq 0$. Then $u_t+\beta u_x=0$. The implicit representation gives
    $$X(s,t)=s+tf''(u_0(s))=s+\beta t.$$
    Thus, for all $x\in \R$, $t>0$,
    $$u(x,t)=u(x+\beta t).$$
\end{example}
\begin{example}
    Consider the Burgers' equation $f(v)=\frac{1}{2}v^2$. Then the PDE is given by  $u_t+uu_x=0$. We take the initial condition
    $$u_0=\begin{cases}
        1,&x\leq 0 \\
        1-x,&0<x<1 \\
        0,&x\geq 1.
    \end{cases}$$
    From the implicit representation,
    $$X(s,t)=s+tf'(u_0(s))=s+tu_0(s)=\begin{cases}
        s+t,&s\leq 0 \\
        s+t(1-s)&0<s<1 \\
        s,&s\geq 1.
    \end{cases}$$
    Drawing the characteristics, we see that at $s=1$ the solution becomes discontinuous. For $s\geq 1$ the characteristics are vertical lines while for $s<1$, the characteristics are linear with slope $1$. Thus, the characteristics will intersect for $s\geq 1$ and are traced back to two different points. Therefore, we see that $T^*=1$ and that smoothness is lost in finite time. We refer to this as the solution developing a \textbf{shock}.
    Solving on a case by case basis, we obtain
    $$S(x,t)=\begin{cases}
        x-t & x\leq t \\
        \frac{x-t}{1-t},&t<x<1 \\
        x,&x\geq 1
    \end{cases}$$
    where $u(x,t)=u_0(S(x,t))$. Thus,
    $$u(x,t)=\begin{cases}
        1,&x\leq t \\
        1-\frac{x-t}{1-t},&t<x<1 \\
        0,&x\geq 1.
    \end{cases}$$
\end{example}
\begin{theorem}[Rankine-Hugoniot Speed]
    The speed of a shock is
    $$\frac{f(u_R)-f(u_L)}{u_R-u_L}$$
    where $u_R$ is $u$ on the right of the shock and $u_L$ is $u$ on the left of the shock.
\end{theorem}
\subsubsection{Weak Solutions}
In order to make sense of solutions that are not $C^1(\R)$, because either the initial data is not $C^1$ or smoothness is lost at some finite time $T^*$, we now introduce the notion of weak solutions. A weak formulation is obtained by testing the equation with smooth test functions that are compactly supported in $\R\times \R^+$, say $\phi \in C^{1}_0(\R\times \R^+)$.
\begin{definition}
    We say that $u\in L^{\infty}_{\rm loc}(\R\times \R^+)$ is a weak solution if
    $$-\int_{\R^+}\int_{\R}(u\phi_t+f(u)\phi_x)dxdt-\int_{\R}u_0(x)\phi(x,0)dx=0,\quad \forall\:\phi\in C^1_0(\R\times \R^+).$$
\end{definition}
The problem with this definition is that there is no uniquenss.
\begin{example}
    Let $u_0(x)=\begin{cases}
        0,&x\leq 0 \\
        1,&x>0
    \end{cases}$ in Burger's equation. Indeed, we have that
    $$X(s,t)=s+u_0t=\begin{cases}
        s,&s\leq 0 \\
        s+t,s>0
    \end{cases}.$$
    Then by observing the characteristics, we see that there is an empty region in which we have no information from characteristics. This produces two different weak solutions.
    \begin{enumerate}
        \item Solution 1: Shock. We place an artificial shock in the empty region that imposes an artifical barrier between the values $0$ and $1$. Everything to the right of the shock is $1$ and everything to the left is $0$. Using the formula for shock speed, we have that $x-\frac{1}{2}t$ is the shock line.
        \item Solution 2: Rarefaction. We impose a boudary with $u_2(x)=x/t$ for $0\leq x\leq t$ so that the solution changes smoothly from $0$ to $1$. This is physically valid.
    \end{enumerate}
\end{example}
Thus, we extend our weak solutions to a notion of entropy solutions.
\begin{theorem}
    For $f\in \operatorname{Lip}(\R;\R)$,$u_0\in L^{\infty}(\R)$, there is a unique entropy solution that is both a weak solution and satisfies
    $$-\int_{\R^+}\int_{\R}(\eta(u)\phi_++q(u)\phi_{x})dxdt-\int_{\R}\eta(u_0)\phi(\cdot,0)dx\leq 0,\quad \forall\:\phi\in C_0^1(\R\times \R^+;\R^+)$$
    and all entropy pairs $(\eta,q)$. In other words, $\partial_{t}\eta(u)+\partial_{x}q(u)\leq 0$ in the sense of distributions.
\end{theorem}

\subsection{Lecture 3: Wave Equation}
\subsubsection{Problem Statement}
The goal of this lecture is to investigate the following nonlinear partial differential equation. We want to find $u:\R\times \R^+\to \R$ so that
\begin{align*}
    \partial_{tt}-c^2\partial_{xx}u&=0,\quad x\in \R,x\in \R^+ \\
    u(x,0)&=f(x),\quad x\in \R \\
    u_{t}(x,0)&=g(x),\quad x\in \R
    \lim_{x\to \pm \infty}(u(x,t)-f(x))&=0.
\end{align*}
Here $f,g:\R\to \R$ are the initial data and $c$ is the \textbf{wave speed}. This PDE is referred to as the \textbf{wave equation} and it is a hyperbolic problem.
In this lecture we construct a solution to this problem using the Fourier transform
technique.
\subsubsection{Fourier Transform}
\begin{definition}
    Let $f\in L^1(\R)$. We define the Fourier transform of $f$, denoted $\mathcal{F}(f):\R\mapsto \mathbb{C}$, such that
    $$\mathcal{F}(f)(\omega)=\frac{1}{2\pi}\int_{\R}f(x)e^{i\omega x}dx.$$
\end{definition}
Indeed, this definition makes sense since
$$\left|\int_{\R}f(x)e^{i\omega x}dx\right|\leq \int_{\R}|f(x)||e^{i\omega x}|dx=\|f\|_{L^1(\R)}<\infty.$$
\begin{definition}
    Let $f\in L^1(\R)$. We define the inverse Fourier transform of $f$, denoted $\mathcal{F}^{-1}(f):\R\mapsto \mathbb{C}$ such that
    $$\mathcal{F}^{-1}(f)(\omega)=\int_{\R}f(x)e^{i\omega x}dx.$$
\end{definition}
\begin{remark}
    Note that many authors will often swap these definitions. Also, many will change the constant from $1/2\pi$ and $1$ to $1/\sqrt{2\pi}$ in both so that the Fourier transform is unitary. All of these are simply conventions.
\end{remark}
\begin{theorem}
    Let $f\in L^1(\R)\cap C^1(\R)$. Then, $\mathcal{F}^{-1}[\mathcal{F}(f)](x)=f(x)$ for all $x\in \R$. If $f$ is discontinuous at $x_0$ but piecewise $C^1$, then
    $$\mathcal{F}^{-1}[\mathcal{F}(f)](x)=\frac{f(x_0^{-})+f(x_0^{+})}{2}.$$
\end{theorem}
\begin{example}
    Here are examples of Fourier transform of some standard functions.
    $$\mathcal{F}(e^{-\alpha|x|})(\omega)=\frac{1}{\pi}\frac{\alpha}{\omega^2+\alpha^2},\quad \mathcal{F}\left(\frac{2}{x^2+\alpha^2}\right)(\omega)=e^{-\alpha|\omega|}.$$
    $$\mathcal{F}(e^{-\alpha x^2})(\omega)=\frac{1}{\sqrt{4\pi \alpha}}e^{-\frac{\omega^2}{4\alpha}}.$$
    $$\mathcal{F}(H(x)e^{-\alpha x})(\omega)=\frac{1}{2\pi}\frac{1}{\alpha-i\omega}.$$
\end{example}
\begin{theorem}
    Let $f\in L^1(\R)$ and assume also that $\partial_{x}f\in L^1(\R)$. Then,
    $$\mathcal{F}(\partial_x f)(\omega)=-i\omega \mathcal{F}(f)(\omega),\quad \forall\:\omega\in \R.$$
    Moreover,
    if $f^{(n)}\in L^1(\R)$, then
    $$\mathcal{F}(f^{(n)})(\omega)=(-i\omega)^{n}\mathcal{F}(f)(\omega).$$
\end{theorem}
\begin{remark}
    Let $f:\R\times \R^+\to \R$. Assume that for all $t\in \R^+$, $f(\cdot,t)\in L^1(\R)$ and $\partial_{t}f(\cot,t)\in L^1(\R)$. Then,
    $$\mathcal{F}(\partial_{t}f(\cdot,t))(\omega)=\partial_{t}\mathcal{F}(f(\cdot,t))(\omega).$$
\end{remark}
\begin{lemma}
    Let $f\in L^1(\R)$ and $\beta\in \R$. Then,
    $$\mathcal{F}(f(x-\beta))(\omega)=e^{i\beta \omega}\mathcal{F}(f)(\omega).$$
\end{lemma}
We now introduce the notion of convolution product.
\begin{definition}
    Let $f,g\in L^1(\R)$. We define the function $f\star g$, called the convolution product of $f$ and $g$, by
    $$(f\star g)(x)=\int_{\R}f(y)g(x-y)dy.$$
\end{definition}
\begin{lemma}
    For all $f,g\in L^1(\R)$, $f\star g=g\star f$.
\end{lemma}
\begin{theorem}
    Let $f,g\in L^1(\R)$. Then,
    $$\mathcal{F}(f\star g)=2\pi \mathcal{F}(f)\mathcal{F}(g).$$
\end{theorem}
\subsubsection{The d'Alembert Formula}
Let us take the Fourier transform of the wave equation. We have
$$\mathcal{F}[\partial_{tt}u]-c^2\mathcal{F}[\partial_{xx}u]=0$$
which implies from our results in the previous section that
$$\partial_{tt}\mathcal{F}[u]+c^2\omega^2\mathcal{F}[u]=0.$$
This is an easy ODE to solve and we have that
$$\mathcal{F}[u](\omega,t)=A(\omega)e^{i\omega ct}+B(\omega)e^{-i\omega ct}$$
for constants $A(\omega),B(\omega)$. Fourier transforming the PDE data, we have $\mathcal{Fu}(\omega,0)=\mathcal{F}[f](\omega)$ and similarly for $g$. Thus, we obtain
$$A(\omega)+B(\omega)=\mathcal{F}[f](\omega)$$
$$i\omega c(A(\omega)-B(\omega))=\mathcal{F}[g](\omega).$$
This implies
$$A(\omega)=\frac{1}{2}\mathcal{F}[f](\omega)+\frac{1}{2i\omega c}\mathcal{F}[g](\omega),$$
$$B(\omega)=\frac{1}{2}\mathcal{F}[f](\omega)-\frac{1}{2i\omega c}\mathcal{F}[g](\omega).$$
Thus,
$$\mathcal{F}[u](\omega,t)=\left(\frac{1}{2}\mathcal{F}[f](\omega)+\frac{1}{2i\omega c}\mathcal{F}[g](\omega)\right)e^{i\omega ct}+\left(\frac{1}{2}\mathcal{F}[f](\omega)-\frac{1}{2i\omega c}\mathcal{F}[g](\omega)\right)e^{-i\omega ct}.$$
Note from the shift lemma that
$$\mathcal{F}[f](\omega)e^{i\omega ct}+\mathcal{F}[f](\omega)e^{-i\omega ct}=\mathcal{F}[f(x-ct)+f(x+ct)](\omega).$$
Let us define $G(x)=\int_{0}^{x}g(\xi)d\xi$. Then $\partial_{x}G(x)=g(x)$ and $-i\omega\mathcal{F}[G](\omega)=\mathcal{F}[g](\omega)$. This shows that
$$\frac{1}{i\omega}\mathcal{F}(g)(\omega)e^{i\omega ct}-\frac{1}{i\omega}\mathcal{F}(g)(\omega)e^{-i\omega ct}=-\mathcal{F}[G](\omega)e^{i\omega ct}+\mathcal{F}[G](\omega)e^{-i\omega ct}$$
and by the shift lemma,
$$\frac{1}{i\omega}\mathcal{F}(g)(\omega)e^{i\omega ct}-\frac{1}{i\omega}\mathcal{F}(g)(\omega)e^{-i\omega ct}=-\mathcal{F}[G(x-ct)+G(x+ct)](\omega).$$
Putting everything together,
\begin{align*}
    \mathcal{F}(u)&=\mathcal{F}\left(\frac{1}{2}(f(x-ct)+\frac{1}{2}f(x+ct))+\frac{1}{2c}(G(x+ct)-G(x-ct))\right) \\
    &=\mathcal{F}\left(\frac{1}{2}(f(x-ct)+\frac{1}{2}f(x+ct))+\frac{1}{2c}\int_{x-ct}^{x+ct}g(\xi)d\xi\right).
\end{align*}
Taking the inverse Fourier transform, we have established the following result.
\begin{theorem}
    The unique weak solution to the wave equation is
    $$u(x,t)=\frac{1}{2}(f(x-ct)+f(x+ct))+\frac{1}{2c}\int_{x-ct}^{x+ct}g(\xi)d\xi.$$
\end{theorem}
To convince ourselves that when $f$ and $g$ are smooth, this solution is unique, we can use an energy approach. We multiply by $\partial_{t}u$ so that
\begin{align*}
    0&=\int_{\R}[u_{tt}u_{t}-c^2u_{xx}u_{t}]dx\\
    &=\frac{d}{dt}\int_{\R}\frac{u_{t}^2}{2}dx-c^2\int_{\R}u_{xx}u_{t}dx \\
    &=\frac{d}{dt}\int_{\R}\frac{u_{t}^2}{2}dx+c^2\int_{\R}u_{x}u_{tx}dx-\underbrace{c^2u_{x}u_{t}\Bigg|_{-\infty}^{\infty}}_{0\text{ by decay}} \\
    &=\frac{d}{dt}\int_{\R}\frac{u_{t}^2}{2}dx+c^2\frac{d}{dt}\int_{\R}\frac{u_{x^2}}{2}dx \\
\end{align*}
so we obtain that
$$\frac{d}{dt}[\|u_t\|^2_{L^2(\R)}+\|u_{x}\|^2_{L^2(\R)}]=0.$$
The quantity inside of the derivative is referred to energy and we see that
$$E(t)=E(0)=\int_{\R}[u_t(x,0)^2+c^2u_x(x,0)^2]dx=\int_{\R}[g(x)^2+c^2f(x)^2]=\|g\|^2_{L^2(\R)}+c^2\|f\|^2_{L^2(\R)}.$$
Now if $u_1$ and $u_2$ are solutions corresponding to the data $(f,g)$, then the energy implies for $w=u_1-u_2$, $\|w_{t}(\cdot,t)\|^2_{L^2(\R)}=0$ and $\|w_{x}(\cdot,t)\|^2_{L^2(\R)}=0$. This implies that $w$ is constant in space and time. The initial condition $w(x,\cdot)=0$ implies $w=0$ identically.

\section{Day 4: Finite Difference Methods}
\subsection{Lecture 1: Finite Differences Approximation}
\subsection{Lecture 2: Time-Domain Problems}
\subsection{Lecture 3: Time-Domain Problems Cont.}

\section{Day 5: Finite Element Methods}
\subsection{Lecture 1: Preliminaries}
\subsubsection{Motivation}
Consider the following ODE:
$$-(p(x)u(x))'=f(x),\quad x\in (0,1)$$
where $p,f:(0,1)\to \R$ are given with $p(x)>0$ for all $x\in (0,1)$ and $u:(0,1)\to \R$ an unknown function to be found.
\begin{example}
    A typical example modeled by the above is the equilibrium temperature
$u$ of a rod represented by the interval $(0, 1)$, given a heat conductivity $p$ and a
heat source $f$.
\end{example}
The ODE does not uniquely determine the solution $u$. In addition, we need to include boundary conditions. We shall focus on the Dirichlet boundary conditions
$$u(0)=\alpha,\quad u(1)=\beta,$$
where $\alpha,\beta\in\R$ are given.
\begin{example}
    In the setting of the previous example, Dirichlet conditions impose a fixed temperature at the ends of the rod. Neumann conditions impose temperature fluxes at the end.
\end{example}

\begin{definition}
    Let $C^0[0,1]$ be the space of continuous functions on $[0,1]$, and, for $m\geq 1$, $C^[0,1]$ the space of functions $f$ such that $f^m\in C^0[0,1]$.
\end{definition}
\begin{remark}
    The ODE appears to require $p\in C^1(0,1)$, $f\in C^0[0,1]$, and $u\in C^2(0,1)$. However, the energy is given by
    $$\frac{1}{2}\int_{0}^{1}p'(x)|u'(x)|^2dx=\int_{0}^[1]f(x)u(x)dx$$
    and requires less regularity. What is the expected regularity of $u$? Using weak derivatives, we establish that not even $u\in C^0[0,1]$ is necessary for the system to have finite energy.
\end{remark}
In view of the previous remark, is it possible to construct numerical schemes
that do not require “smooth” data and solutions?

\subsubsection{Weak Derivatives}
\begin{definition}
    We say that a function $f:(0,1)\to \R$ is \textbf{square-integrable} in $(0,1)$ if it is integrable and
    $$\int_{0}^{1}|f(x)|^2dx<\infty.$$
    The set of all such function is denoted $L^2(0,1)$, that is,
    $$L^2(0,1)=\left\{f:(0,1)\to \R \text{ integrable}\:|\:\int_{0}^{1}|f(x)|^2dx<\infty\right\}.$$
    It is a Hilbert space equipped with the inner product and norm,
    $$(f,g)_{L^2(0,1)}=\int_{0}^{1}f(x)g(x)dx,\quad (f,f)=\|f\|_{L^2(0,1)}^{1/2}.$$
\end{definition}
\begin{lemma}
    Let $f,g\in L^2(0,1)$. Then,
    $$(f,g)_{L^2(0,1)}\leq \|f\|_{L^2(0,1)}\|g\|_{L^2(0,1)}.$$
\end{lemma}
\begin{example}
    The set $L^2(0,1)$ contains discontinuous functions. For example, consider
    $$f(x)=\begin{cases}
        -1&0<x<\frac{1}{2},\\
        \pi&x=\frac{1}{2},\\
        1&\frac{1}{2}<x<1.
    \end{cases}$$
    Then,
    $$\int_{0}^[1]f(x)dx=\int_{0}^{1/2}(-1)dx+\int_{1/2}^{1}1dx=0$$
    and
    $$\int_{0}^{1}|f(x)|^2=\int_{0}^{1}1dx=1<\infty$$
    so $f\in L^2(0,1)$. Also notice that sets of measure zero (i.e. single points) do not contribute to the integral.
\end{example}
Let $v\in C^1[0,1]$ and note that for all $w\in C^1[0,1]$ with $w(0)=w(1)=0$, integration by parts produces
$$\int_{0}^{1}v'(x)w(x)=-\int_{0}^{1}v(x)w'(x).$$
More compactly,
$$(v',w)_{L^2(0,1)}=-(v,w')_{L^2(0,1)}.$$
We ofte write $C^1_0[0,1]$ for continuously differentiable functions that are zero on the boundary.
This justifies the following definition.
\begin{definition}
    Let $v\in L^2(0,1)$. We say that $v$ has a \textbf{weak derivative} in $L^2(0,1)$ if there exists $\phi\in L^2(0,1)$ such that
    $$(\phi,w)_{L^2(0,1)}=-(v,w')_{L^2(0,1)},\quad \forall\:w\in C^1_0[0,1].$$
    In this case, we write $\phi=v'$.
\end{definition}
We accept the following facts about weak derivatives:
\begin{itemize}
    \item Changing the value of a function at one point does not change its weak
derivative;
\item If a weak derivative exists, it must be unique (up to the value at a finite
number of points), and so generates an equivalence class.
\end{itemize}
The uniqueness of weak derivative implies that if $v\in C^1[0,1]$ then the
standard derivative is also the weak derivative.
\begin{example}
    Consider the function
    $$v(x)=\begin{cases}
        2x&0<x\leq \frac{1}{2}, \\
        2-2x&\frac{1}{2}<x<1, \\
        0& \text{otherwise}.
    \end{cases}$$
    This function is not in $C^1[0,1]$. However, for any $w\in C^1_0[0,1]$, we have
    \begin{align*}
        \int_{0}^{1}v(x)w'(x)&=2\int_{0}^{1/2}xw'(x)dx+2\int_{1/2}^{1}w'(x)-2\int_{1/2}^{1}xw'(x)dx \\
        &=2\left(-\int_{0}^{1/2}w(x)dx+xw(x)\Bigg|_{0}^{1/2}\right)+2w(1)-2w(1/2)-2\left(-\int_{1/2}^{1}w(x)dx+xw(x)\Bigg|_{1/2}^{1}\right) \\
        &=-2\left(\int_{0}^{1}w(x)dx-\int_{1/2}^{1}w(x)dx\right)+2\frac{1}{2}w(1/2)-2w(1/2)-2(w(1)-1/2w(1/2)) \\
        &=-2\left(\int_{0}^{1}w(x)dx-\int_{1/2}^{1}w(x)dx\right) \\
        &=-\left(\int_{0}^{1/2}(2)w(x)dx+\int_{1/2}^{1}(-2)w(x)dx\right).
    \end{align*}
    Therefore, we identify
    $$\phi(x)=\begin{cases}
        2&0<x\leq \frac{1}{2} \\
        -2&\frac{1}{2}<x<1
    \end{cases}$$
    so that we have
    $$(v,w')_{L^2(0,1)}=-(\phi,w),\quad \forall\:w\in C^1_0[0,1].$$
    Notice that $\phi\in L^2(0,1)$. Therefore, $\phi$ is the weak derivative of $v$. Notice that $v'$ does not have a weak derivative in $L^2(0,1)$. Indeed, for all $w\in C^1_0[0,1]$,
    $$\int_{0}^{1}\phi(x)v'(x)dx=\int_{0}^{1/2}2v'(x)-\int_{1/2}^{1}2v'(x)=4v(1/2)$$
    and there is no $L^2(0,1)$ function $\psi$ such that
    $$4v(1/2)=-\int_{0}^{1}\psi(x)v(x)dx.$$
    In fact, $\psi(x)=-4\delta_{1/2}(x)\notin L^2(0,1)$, where $\delta_{1/2}(x)$ is the Dirac measure at $1/2$.
\end{example}
\begin{definition}
    We denote by $H^1(0,1)$ the \textbf{Sobolev space} of $L^2(0,1)$ functions having a weak derivative in $L^2(0,1)$, i.e.,
    $$H^1(0,1)=\{v\in L^2(0,1)\:|\:v'\in L^2(0,1)\}.$$
    It is a Hilbert space with the inner product and norm,
    $$(f,g)^2_{H^1(0,1)}=(f,g)^2_{L^2(0,1)}+(f',g')^2_{L^2(0,1)},\quad \|f\|^2_{H^1(0,1)}=(f,f)^{1/2}_{H^1(0,1)}.$$
\end{definition}
Like $L^2(0,1)$, the set $H^1(0,1)$ consists of equivalence classes of functions from their pointwise invariance. We will accept that for every $f\in H^1(0,1)$, there is an extension $\tilde{f}\in C^0[0,1]$, that is, $f=\tilde{f}$ almost everywhere. From now on, when we write $f\in H^1(0,1)$, we mean the continuous representation $\tilde{f}$ so that pointwise values of $f$ are well-defined.
\begin{lemma}
    For $v,w\in H^1(0,1)$, it holds that
    $$\int_{0}^{1}v'(x)w(x)dx=-\int_{0}^{1}v(x)w'(x)+v'(x)w(x)\Bigg|_{0}^{1}.$$
\end{lemma}
\begin{definition}
    The set of functions
    $$H^1_0(0,1)=\{v\in H^1(0,1)\:|\:v(0)=v(1)=0\}$$
    is the subset of $H^1(0,1)$ consisting of functions vanishing at $0$ and $1$.
\end{definition}
Note that if $v$ is smooth and satisfies $v(0)=0$, then
$$v(x)^2-v(0)^2=\int_{0}^{x}(v(s)^2)'ds=2\int_{0}^{x}v(s)v'(s)ds.$$
Therefore,
$$v(x)^2\leq 2\int_{0}^{1}|v(s)v'(s)|ds\leq 2\|v\|_{L^2(0,1)}\|v'\|_{L^2(0,1)}.$$
After integrating from $0$ to $1$, we deduce that
$$\|v\|^2_{L^2(0,1)}\leq 2\|v\|_{L^2(0,1)}\|v'\|_{L^2(0,1)}$$
or equivalently,
$$\|v\|^2_{L^2(0,1)}\leq 2\|v'(0,1).$$
This estimate is known as the \textbf{Poincare inequality} and is more generally true for functions in $H^1_0(0,1)$.
\begin{lemma}
    For $v\in H^1_0(0,1)$, there exists $C>0$ such that
    $$\|v\|_{L^2(0,1)}\leq C\|v'\|_{L^2(0,1)}.$$
\end{lemma}
\subsubsection{Weak Formulation}
We return to the problem of finding $u\in C^2[0,1]$ satisfying
$$-(p(x)u'(x))'=f(x),\quad x\in (0,1),\quad u(0)=u(1)=0.$$
We assume that $p\in L^2(0,1)$ is such that $0<P_{\rm min}\leq p(x)\leq P_{\rm max}$ a.e. for some $0<P_{\rm min}\leq P_{\rm max}<\infty$ and that $f\in L^2(0,1)$. Notice that in particular f is not necessarily continuous, which therefore requires
us to give a different meaning to the ODE.
\begin{remark}
    For the case with general Dirichlet boundary conditions $u(0)=\alpha$, $u(1)=\beta$, we set $u_0=\alpha+(\beta-\alpha)x$ so that $\tilde{u}=u-u_0$, satisfies the ODE with zero boundary conditions with $f(x)$ replaced by $f(x)+(\beta-\alpha)p'(x)$.
\end{remark}
For now, we assume $u\in C^2[0,1]$ and multiply the ODE by $v\in H^1_0(0,1)$ and integrate by parts
$$\int_{0}^{1}p(x)u'(x)v'(x)dx-p(x)u'(x)v(x)\Bigg|_{0}^{1}=\int_{0}^{1}f(x)v(x)dx.$$
Because $v(0)=v(1)=0$,
$$\int_{0}^{1}p(x)u'(x)v'(x)dx=\int_{0}^{1}f(x)v(x)dx,\quad \forall\:v\in H^1_0(0,1).$$
Notice that from Cauchy-Schwarz and the assumptions on $p$ and $f$, we have
$$\left|\int_{0}^{1}p(x)u'(x)v'(x)dx\right|\leq P_{\rm max}\|u\|_{H^1(0,1)}\|v\|_{H^1(0,1)}<\infty$$
and
$$\left|\int_{0}^{1}f(x)v(x)dx\right|\leq \|f\|_{L^2(0,1)}\|v\|_{H^1(0,1)}<\infty.$$
This justifies the following definition.
\begin{definition}
    The \textbf{weak formulation} of the ODE is to find $u\in H^1_0(0,1)$ such that
    $$\int_{0}^{1}p(x)u'(x)v'(x)dx=\int_{0}^{1}f(x)v(x)dx,\quad \forall\:v\in H^1_0(0,1).$$
    This $u\in H^1_0(0,1)$ is said to be a \textbf{weak solution} to the ODE.
\end{definition}
The next result state the existence and uniqueness of weak solutions to the
ODE.
\begin{lemma}[Lax-Milgram]
    The weak formulation has a unique solution $u\in H^1_0(0,1)$. It satisfies the stability estimate
    $$\|u\|_{H^1(0,1)}\leq 2P^{-1}_{\rm min}\|f\|_{L^2(0,1)}.$$
\end{lemma}
\subsection{Lecture 2: Finite Elements}
\subsubsection{Discretization}
Our goal is to replace our infinite dimension space $H^1_0(0,1)$ in the weak formulation by a finite dimensional approximation space to compute a solution. Let
$$0\leq x_0<x_1<\cdots<x_{N+1}=1$$
be an partition of $[0,1]$. For each $i=1,\ldots,N$, we define the "hat" function
$$\phi_i(x)=\begin{cases}
    \frac{x-x_{i-1}}{x_i-x_{i-1}}&x\in [x_{i-1},x_i],\\
    \frac{x_{i+1}-x}{x_{i+1}-x_i}&x\in [x_{i},x_{i+1}],\\
    0&\text{otherwise}.
\end{cases}$$
Each $\phi_i$ is piecewise linear and therefore in $H^1_0(0,1)$. Moreover, $\phi_i(x_j)=\delta_{ij}$ which implies that $\{\phi_i\}_{i=1}^{N}$ are linerly independent. To see this, assume that for some $\{\alpha_i\}_{i=1}^{N}\in \R$ there holds
$$\sum_{i=1}^{N}\alpha_i\phi_i(x)=0,\quad \forall\:x\in [0,1].$$
Then, for $j\in [N]$, we have
$$0=\sum_{i=1}^{N}\alpha_i \phi_i(x_j)=\sum_{i=1}^{N}\alpha_i\delta_{ij}=\alpha_j.$$
Therefore, this implies $\alpha_j=0$ for all $j$ and $\{\phi_i\}$ are linearly independent. Let $V_N=\operatorname{span}(\phi_1,\ldots,\phi_N)\subset H^1_0(0,1)$, i.e.,
$$V_N=\left\{\sum_{i=1}^{N}\alpha_i\phi_i(x)\:|\:\alpha_i\in \R,\:i=1,\ldots,N\right\}.$$
From linear independence, $V_N$ has dimension $N$. Define
$$W_N=\left\{v\in C^0[0,1]\:|\:v(0)=v(1)=0,\:\:v|_{[x_i,x_{i+1}]} \text{ is linear}, \:i=0,\ldots,N\right\}.$$
Then, $V_N\subset W_N$ because each $\phi_i\in W_N$. Moreeover, any $w\in W_N$ can be written as
$$w(x)=\sum_{i=1}^{N}w(x_i)\phi_i(x)$$
which implies $w\in V_N$. Thus, $W_N\subset V_N$ and consequently $V_N=W_N$.
\subsubsection{Linear System Formulation}
We replace $H^1_0(0,1)$ by the finite dimensional subspace $V_N$.
\begin{definition}
    The discrete weak formulation reads: Find $u_N\in V_N$ such that
    $$\int_{0}^{1}p(x)u'_N(x)v'_N(x)dx=\int_{0}^{1}f(x)v_N(x)dx,\quad v_N\in V_N.$$
\end{definition}
As we shall see, a unique solution $u_N\in V_N$ exists to the above problem. This $u_N$ is called the finite element solution. Notice that since the discrete weak formulation holds for all $v_N\in V_N$, it also holds that
$$\int_{0}^{1}p(x)u'_N(x)\phi_j(x)dx=\int_{0}^{1}f(x)\phi_j(x)dx,\quad j=1,\ldots,N.$$
Since $u_N\in V_N$, we have that
$$u_N(x)=\sum_{i=1}^{N}U_i\phi_i(x)$$
for some set of coefficients $\{U_i\}\in \R$. Using this ansatz, we obtain that $u_N$ is the finite element solution if and only if
$$\sum_{i=1}^{N}U_i\int_{0}^{1}p(x)\phi_i'(x)\phi'_j(x)dx=\int_{0}^{1}f(x)\phi_j(x)dx,\quad j=1,\ldots,N.$$
Define the \textbf{stiffness matrix}
$$A=(a_{ij})_{i,j=1}^{N},\quad a_{ij}=\int_{0}^{1}p(x)\phi'_i(x)\phi'_j(x)dx$$
and the vectors
$$F=(F_j)_{j=1}^{N},\quad F_j=\int_{0}^{1}f(x)\phi_j(x),\quad U=(U_j)_{j=1}^{N}.$$
With this notation, $u_N$ is the finite element solution if and only if its coefficients satisfy the linear system
$$AU=F.$$
\begin{remark}
Notice that this system is sparse. Indeed, when $|i-j|>1$, $a_{ij}=1$ so only the three primary diagonals are populated. The use of sparse matrices is critical
because it requires (approximately) the storage of $3N$ doubles instead of $N^2$
doubles. Recall that $1$ double is $8$ bytes or $8\times 10^{-7}$ MB. For $N = 106$
the use of a sparse matrix needs $2.4$ MB while a full matrix requires $800$ GB. Moreover, Gaussian elimination (not used in practice) requires $O(N^3)$ operations for a full matrix. For an $n$-diagonal matrix, only $nN$ operations are needed.
\end{remark}
\begin{remark}
    When $p(x)=1$ and $x_i=ih$ with $h=1/(N+1)$, $A$ is the finite difference matrix up to a scaling factor.
\end{remark}

\subsection{Lecture 3: Well-posedness}
\subsubsection{Existence and Uniqueness}
To show that there is a unique vector $U\in \R^N$ satisfying the linear system $AU=F$, we need to show that $A$ is invertible. We show that $\operatorname{Ker}(A)=\{0\}$. Assume that $AV=0$ for some $V\in \R^N$. Therefore,
$$V^TAV=0 \implies \sum_{i,j=1}^{N}V_ia_{ij}V_j=0.$$
From the definition of $a_{ij}$, we find that
$$\int_{0}^{1}p(x)|v'_N(x)|^2dx=0$$
where $v_N=\sum_{i=1}^{N}V_i\phi_i(x)$. Taking advantage of the assumption on $p$, we deduce that
$$\|v'_N\|_{L^2(0,1)}=0 \implies v_N=0.$$
From linear independence, we obtain that $V=0$ and $A$ is therefore injective. But since $A$ is finite dimensional, it is surjective and therefore invertible.

\subsubsection{Stability}
Taking $v_N=u_N$, we find that
$$\int_{0}^{1}p(x)|u_N'(x)|^2dx=\int_{0}^{1}f(x)u_N(x)dx.$$
From the Cauchy Schwarz inequality and the assumption on $p$,
$$P_{\rm min}\|u'_N\|^2_{L^2(0,1)}\leq \|f\|_{L^2(0,1)}\|u_N\|_{L^2(0,1)}\leq \|f\|_{L^2(0,1)}\|u_N\|_{H^1(0,1)}$$
or
$$\|u_{N}'\|^2_{L^2(0,1)}\leq P^{-1}_{\rm min}\|f\|_{L^2(0,1)}\|u_N\|_{H^1(0,1)}.$$
The Poincare inequality implies that there exists $C>0$ such that
$$\|u_N\|_{L^2(0,1)}\leq \|u'_N\|_{L^2(0,1)}$$
and so
$$\|u_N\|_{H^1(0,1)}\leq CP^{-1}_{\rm min}\|f\|_{L^2(0,1)}.$$
Since in our case $C=2$, we see that this is identical to the Lax-Milgram lemma
$$\|u\|_{H^1(0,1)}\leq 2P^{-1}_{\rm min}\|f\|_{L^2(0,1)}.$$
In particular, for two solutions $u^1_N,u^2_N$ corresponding to $f^1,f^2$, we have
$$\|u^1_N-u^2_N\|_{H^1(0,1)}\leq 2P^{-1}_{\rm mi}\|f^-f^2\|_{L^2(0,1)}$$
and stability.
\subsubsection{Convergence}
For any $v_N\in V_N$, the weak solution satisfies
$$\int_{0}^{1}p(x)u'(x)v'_N(x)dx=\int_{0}^{1}f(x)v_N(x)dx$$
and the finite element solution satisfies
$$\int_{0}^{1}p(x)u'_N(x)v'_N(x)dx=\int_{0}^{1}f(x)v_N(x)dx.$$
By subtracting, we see that
$$\int_{0}^{1}p(x)(u'(x)-u'_N(x))v'_N(x)dx=0,\quad \forall\:v_N\in V_N.$$
This is the \textbf{Galerkin orthogonality}. The orthogonality refers to the fact that the error in the weak derivative of the finite element solution is in the orthogonal complement of $V_N$. In view of this, we compute
\begin{align*}
    \|u'-u'_N\|^2_{L^2(0,1)}&\leq \frac{1}{P_{\rm min}}\int_{0}^{1}p(x)|u'(x)-u'_N(x)|^2dx \\
    &=\frac{1}{P_{\rm min}}\int_{0}^{1}p(x)(u'(x)-u'_N(x))(u'(x)-u'_N(x))dx  \\
    &=\frac{1}{P_{\rm min}}\int_{0}^{1}p(x)(u'(x)-u'_N(x))(u'(x)-v'_N(x))dx  \\
    &\leq \frac{P_{\rm max}}{P_{\rm min}}\|u'-u'_N\|^2_{L^2(0,1)}\|u'-v'_N\|^2_{L^2(0,1)}
\end{align*}
and so
$$\|u'-u'_N\|_{L^2(0,1)}\leq \frac{P_{\rm max}}{P_{\rm min}} \min_{v_N\in V_N}\|u'-v'_N\|^2_{L^2(0,1)}.$$
This is known as the \textbf{best approximation property} as it says that the finite element solution is the best approximation in the chosen space $V_N$.

We will now show convergence. For now, assume that $u\in H^2(0,1)$. Define the linear interpolant of $u$
$$I_Nu(x)=\sum_{i=1}^{N}u(x_i)\phi_i(x).$$
Notice that $u(x_i)=I_Nu(x_i)$ for $i=0,\ldots,N+1$. Define $e(x)=u(x)-I_Nu(x)$ so that $e(x_i)=0$ for all $i=0,\ldots,N+1$. We apply Rolle's theorem to guarantee the existence of $\xi_j\in (x_j,x_{j+1})$ for $j=0,\ldots,N$ such that $e'(\xi_j)=0$. For $x\in (x_j,x_{j+1})$, the Fundamental Theorem of Calculus implies
$$e'(x)=\int_{\xi_j}^{x}e''(s)ds=\int_{\xi_j}^{x}u''(s)ds.$$
From Cauchy-Schwarz,
$$e'(x)^2\leq |x-\xi_j|\int_{\xi_j}^{x}(u''(s))^2ds\leq |x_{j+1}-x_j|\int_{x_j}^{x_{j+1}}(u''(s))^2ds$$
and by integrating,
$$\int_{x_j}^{x_{j+1}}e'(x)^2\leq \max_{j=0,\ldots,N}|x_{j+1}-x_j|^2\int_{x_j}^{x_{j+1}}(u''(s))^2ds.$$
After summing over $j=0,\ldots,N+1$,
$$\|u'-(I_Nu)'\|_{L^2(0,1)}\leq \max_{j=0,\ldots,N}|x_{j+1}-x_| \|u''\|_{L^2(0,1)}.$$
Returning to $\|u'-u'_N\|_{L^2(0,1)}$, the best approximation property yields
$$\|u'-u'_N\|_{L^2(0,1)}\leq \frac{P_{\rm max}}{P_{\rm min}}\|u''\|_{L^2(0,1)}\max_{j=0,\ldots,N}|x_{j+1}-x_j|.$$
The right side tends to $0$ whenever $\max_{j=0,\ldots,N}|x_{j+1}-x_j|\to 0$. When the subdivision is uniform, that is $x_i=i/(N+1)$, we have
$$\|u'-u'_N\|_{L^2(0,1)}\leq \frac{P_{\rm max}}{P_{\rm min}}\|u''\|_{L^2(0,1)}\frac{1}{N+1}\to 0$$
as $N\to \infty$.

\section{Project}
\subsection{Preliminaries}
Let $D\subset \R^3$. We care about studying the radiation $\Psi:D\times \mathbb{S}^2\to \R$, $(x,\Omega)\mapsto \Psi(x,\Omega)$. In particular, we want to find $\Psi$ such that
$$\Omega\cdot \nabla_{X}\Psi(x,\Omega)+\sigma^t\Psi(x,\Omega)=\frac{\Sigma^s}{|\mathbb{S}^2|}\int_{\mathbb{S}^2}\Psi(x,\Omega')d\Omega'+q(x),\quad \text{in }\Omega\times \mathbb{S}^2.$$
$$\Psi(x,\Omega)=\alpha^{\partial}(x),\quad \text{on }\{x\in \partial D,\:\Omega\in \mathbb{S}^2\:|\:n_{x}\cdot \Omega<0\}.$$
Here, $\sigma^t$ is the total cross section, $\sigma^s$ is the scattering cross section, and $\sigma^a=\sigma^t-\sigma^s$ the absorption cross section. The source $q$ occurs from the physical nature of the problem, for example black-body radiation, in which the problem becomes couple with additional PDE constraints on $q$. We say that the Neumann condition is isotropic if there is no dependence on $\Omega$. These problems are relevant to study for neutron scattering and nuclear fusion.

To numerically obtain a solution, we use the \textbf{discrete ordinates method}. Let us establish a quadrature rule over $\mathbb{S}^2$ with weights $\{w_{l}\}_{l\in \mathcal{L}}$ such that $\sum_{l}w_l=1$. Then, we want to find $\{\psi^l\}_{l\in \mathcal{L}}$ such that
$$\Omega_{l}\cdot \nabla \psi^{l}+\sigma^t(x)\psi^{l}=\sigma^{s}(x)\sum_{k}\omega_k\psi^{k}+q(x),\quad \forall\:l\in \mathbb{L}$$
where $0\leq \sigma^s\leq \sigma^t$ for all $x\in D$. We will assume that $\sigma^s,\sigma^t,q$ are piecewise constant on each cell in our mesh.

For now, let us simplify this problem to the interval $[a,b]$. Fix $\mu\neq 0$. If $\mu>0$, we have an inflow problem and prescribe a left boundary. If $\mu<0$, we have an outflow problem and prescribe a right boundary. We want to find $u:[a,b]\to \R$ such that
$$\mu u'+\sigma^tu=q$$
$$u(a)=\alpha.$$
Multiplying by a suitable test function $v$ and integrating, we have
$$\int_{a}^{b}\mu u'v+\sigma^tuv=\int_{a}^{b}qvdx.$$
Hence, the discrete weak formulation is to find $u_h\in V_h$ such that
$$\int_{a}^{b}\mu u_hv_h+\sigma^tu_hv_h=\int_{a}^{b}qv_hdx,\quad \forall\:v_h\in V_h.$$
Taking $u_h=\sum_{i=0}^{N-1}u_i\phi_i$ and testing with $v_h=\phi_j$, we have
$$\sum_{i=1}^{n}u_i\int_{a}^{b}(\mu\phi_i'\phi_j+\sigma^t\phi_i\phi_j)dx=\int_{a}^{b}q\phi_jdx,\quad i=0,\ldots,N-1.$$
Define
$$(T)_{ij}=\int_{a}^{b}(\phi_i'\phi_j+\sigma^t\phi_i\phi_j),\quad b_i=\int_{a}^{b}q\phi_jdx.$$
This is an $N-1\times N-1$ system. To impose the boundary condition, we artifically impose it into the linear system through an additional row. From here, we simply solve $Tu=b$ for the coefficient vector $u$.

Solving this system, we see that we have stability issues from spurious oscillations in the solution. This occurs from applying a naive Galerkin scheme to first order PDE. To fix this, we utilize the Streamline-Upwind Petrov-Galerkin (SUPG) scheme. We instead consider the weak formulation: Find $u\in V$ such that
$$\int_{a}^{b}(\mu u'+\sigma^tu)(v+\tau \mu v')dx=\int_{a}^{b}q(v+\tau\mu v')dx,\quad \forall v\in V.$$
The upwind approach comes from modifying the test function term with its derivative. Moreover, the method is Petrov-Galerkin as the test functions lie in a different space $V$. The coefficient $\tau$ is defined by
$$\tau=\xi \operatorname{max}(|\mu|h^{-1},\sigma^t)^{-1}$$
where $\xi$ is a tuning parameter. Generally, we take $\xi\approx 1$ and $\tau=\tau_k$ that varies over each cell.

We now have a code that can solver transport for one fixed problem. Suppose now that we project the sphere onto $[-1,1]$ with a quadrature set $\mathcal{L}$ with weights
$$\sum_{l\in \mathcal{L}}w_l=1,\quad \sum_{l}\omega_lf(\mu_l)\approx \frac{1}{2}\int_{[-1,1]}f(\mu)d\mu.$$
We want to find $\{\Psi^l_h\}_{l\in \mathcal{L}}\in V_h^{\mathcal{L}}$ (one for every angle) such that
$$t_h^l(\Psi^l_h,\phi)+s_h^l(\Psi^l_h,\phi)+b_h^{l}(\Psi^l_h,\phi)=(\sigma^s\sum_{l}w_l\Psi^l_h,\phi+\tau\mu\phi')+(q,\phi+\tau\mu^l\phi')+b_h^l(\Psi^l_h,\phi),\quad \forall\:\phi \in V_h$$
Define $\Psi_h^{l,*}\in V_h$ that solves
$$t_h^l(\Psi_h^{l,*},\phi)+s_h^l(\Psi_h^{l,*},\phi)+b_h^{l}(\Psi_h^{l,*},\phi)=(q,\phi+\tau\mu^l\phi')+b_h^l(\Psi_h^{l,*},\phi),\quad \forall\:\phi \in V_h.$$
In other words, it solves
$$\mu^lu'+\sigma^tu=q$$
along with the boundary conditions. Define $\Psi_h^{l,0}:V_h\to V_h$ that solves $\mu^lu'+\sigma^{t}u=\sigma^s=\varphi$, i.e,
$$t_h^{l}(\psi_h^{l,0}(\varphi),\Phi)+s_h^{0}(\Psi_h^{l,0}(\varphi),\phi)=(\sigma^2\varphi,\phi+\tau\mu^l\phi').$$
The desired solution
$$\Gamma_h^0(\phi)=(\Psi_h^{1,0},\Psi_h^{2,0},\ldots,\Psi^{|\mathcal{L}|,0}),$$
$$\Gamma_h^*(\phi)=(\Psi_h^{1,*},\Psi_h^{2,*},\ldots,\Psi_h^{|\mathcal{L},*})$$
$$\phi=\sum_{l}w_l(\Psi_h^{l,0}+\Psi_h^{l,*})=\overline{\Psi_h^{l,0}(\phi)}+\overline{\Psi_h^{l,*}}.$$
To summarize, we
\begin{itemize}
    \item Compute $\overline{\Psi_h^{l,*}}$ by applying our previous method for every angle.
    \item We make an initial guess for $\phi^{(0)}$.
    \item Then we compute
    $$\phi^{(n+1)}=\overline{\Psi{n}^{l,0}(\phi^{(n)})}+\overline{\Psi^{l,*}}$$
    \item Finally, we construct the actual intensity
    $$\Psi_h^{l}=\Psi^{l,0}(\phi_{\rm soln})+\Psi^{l,*}.$$
\end{itemize}

Going back to our original problem:
$$\Omega\cdot \nabla_x \Psi+\sigma^t(x)\Psi=\frac{\sigma^t(x)}{|\mathbb{S}^2|}\int_{\mathbb{S}^2}\psi(x,\Omega')d\Omega'+q(x)$$
where
$\Psi=\alpha^{\partial}$ on the inflow boundary. Let $\epsilon>0$. We can non-dimensionalize this PDE into the form
$$\Omega\cdot \nabla_x \Psi+\sigma\left(\epsilon+\frac{1}{\epsilon}\right)\Psi=\frac{\sigma}{\epsilon|\mathbb{S}^2|}\int_{\mathbb{S}^2}\psi(x,\Omega')d\Omega'+\epsilon \overline{q}(x)$$
where $\Psi=\epsilon \overline{alpha}^{\partial}$ on the inflow boundary. We have the uniform limit
$$\Psi=\Psi^{(0)}+\epsilon \Psi^{(1)}+\epsilon^2\Psi^{(2)}+\cdots$$
where $\Psi^{(0)}$ is isotropic (no dependence on $\Omega$) and satisfies the PDE
$$-\nabla \left(\frac{1}{3\sigma^{s}}\nabla \Psi^{(0)}\right)+\sigma^{a}\Psi^{(s)}=q$$
where $\Psi^{(0)}=\alpha^{\partial}$ on $\partial D$.

\end{document}
